% Options for packages loaded elsewhere
\PassOptionsToPackage{unicode}{hyperref}
\PassOptionsToPackage{hyphens}{url}
%
\documentclass[
]{book}
\usepackage{amsmath,amssymb}
\usepackage{lmodern}
\usepackage{iftex}
\ifPDFTeX
  \usepackage[T1]{fontenc}
  \usepackage[utf8]{inputenc}
  \usepackage{textcomp} % provide euro and other symbols
\else % if luatex or xetex
  \usepackage{unicode-math}
  \defaultfontfeatures{Scale=MatchLowercase}
  \defaultfontfeatures[\rmfamily]{Ligatures=TeX,Scale=1}
\fi
% Use upquote if available, for straight quotes in verbatim environments
\IfFileExists{upquote.sty}{\usepackage{upquote}}{}
\IfFileExists{microtype.sty}{% use microtype if available
  \usepackage[]{microtype}
  \UseMicrotypeSet[protrusion]{basicmath} % disable protrusion for tt fonts
}{}
\makeatletter
\@ifundefined{KOMAClassName}{% if non-KOMA class
  \IfFileExists{parskip.sty}{%
    \usepackage{parskip}
  }{% else
    \setlength{\parindent}{0pt}
    \setlength{\parskip}{6pt plus 2pt minus 1pt}}
}{% if KOMA class
  \KOMAoptions{parskip=half}}
\makeatother
\usepackage{xcolor}
\usepackage{color}
\usepackage{fancyvrb}
\newcommand{\VerbBar}{|}
\newcommand{\VERB}{\Verb[commandchars=\\\{\}]}
\DefineVerbatimEnvironment{Highlighting}{Verbatim}{commandchars=\\\{\}}
% Add ',fontsize=\small' for more characters per line
\usepackage{framed}
\definecolor{shadecolor}{RGB}{248,248,248}
\newenvironment{Shaded}{\begin{snugshade}}{\end{snugshade}}
\newcommand{\AlertTok}[1]{\textcolor[rgb]{0.94,0.16,0.16}{#1}}
\newcommand{\AnnotationTok}[1]{\textcolor[rgb]{0.56,0.35,0.01}{\textbf{\textit{#1}}}}
\newcommand{\AttributeTok}[1]{\textcolor[rgb]{0.77,0.63,0.00}{#1}}
\newcommand{\BaseNTok}[1]{\textcolor[rgb]{0.00,0.00,0.81}{#1}}
\newcommand{\BuiltInTok}[1]{#1}
\newcommand{\CharTok}[1]{\textcolor[rgb]{0.31,0.60,0.02}{#1}}
\newcommand{\CommentTok}[1]{\textcolor[rgb]{0.56,0.35,0.01}{\textit{#1}}}
\newcommand{\CommentVarTok}[1]{\textcolor[rgb]{0.56,0.35,0.01}{\textbf{\textit{#1}}}}
\newcommand{\ConstantTok}[1]{\textcolor[rgb]{0.00,0.00,0.00}{#1}}
\newcommand{\ControlFlowTok}[1]{\textcolor[rgb]{0.13,0.29,0.53}{\textbf{#1}}}
\newcommand{\DataTypeTok}[1]{\textcolor[rgb]{0.13,0.29,0.53}{#1}}
\newcommand{\DecValTok}[1]{\textcolor[rgb]{0.00,0.00,0.81}{#1}}
\newcommand{\DocumentationTok}[1]{\textcolor[rgb]{0.56,0.35,0.01}{\textbf{\textit{#1}}}}
\newcommand{\ErrorTok}[1]{\textcolor[rgb]{0.64,0.00,0.00}{\textbf{#1}}}
\newcommand{\ExtensionTok}[1]{#1}
\newcommand{\FloatTok}[1]{\textcolor[rgb]{0.00,0.00,0.81}{#1}}
\newcommand{\FunctionTok}[1]{\textcolor[rgb]{0.00,0.00,0.00}{#1}}
\newcommand{\ImportTok}[1]{#1}
\newcommand{\InformationTok}[1]{\textcolor[rgb]{0.56,0.35,0.01}{\textbf{\textit{#1}}}}
\newcommand{\KeywordTok}[1]{\textcolor[rgb]{0.13,0.29,0.53}{\textbf{#1}}}
\newcommand{\NormalTok}[1]{#1}
\newcommand{\OperatorTok}[1]{\textcolor[rgb]{0.81,0.36,0.00}{\textbf{#1}}}
\newcommand{\OtherTok}[1]{\textcolor[rgb]{0.56,0.35,0.01}{#1}}
\newcommand{\PreprocessorTok}[1]{\textcolor[rgb]{0.56,0.35,0.01}{\textit{#1}}}
\newcommand{\RegionMarkerTok}[1]{#1}
\newcommand{\SpecialCharTok}[1]{\textcolor[rgb]{0.00,0.00,0.00}{#1}}
\newcommand{\SpecialStringTok}[1]{\textcolor[rgb]{0.31,0.60,0.02}{#1}}
\newcommand{\StringTok}[1]{\textcolor[rgb]{0.31,0.60,0.02}{#1}}
\newcommand{\VariableTok}[1]{\textcolor[rgb]{0.00,0.00,0.00}{#1}}
\newcommand{\VerbatimStringTok}[1]{\textcolor[rgb]{0.31,0.60,0.02}{#1}}
\newcommand{\WarningTok}[1]{\textcolor[rgb]{0.56,0.35,0.01}{\textbf{\textit{#1}}}}
\usepackage{longtable,booktabs,array}
\usepackage{calc} % for calculating minipage widths
% Correct order of tables after \paragraph or \subparagraph
\usepackage{etoolbox}
\makeatletter
\patchcmd\longtable{\par}{\if@noskipsec\mbox{}\fi\par}{}{}
\makeatother
% Allow footnotes in longtable head/foot
\IfFileExists{footnotehyper.sty}{\usepackage{footnotehyper}}{\usepackage{footnote}}
\makesavenoteenv{longtable}
\usepackage{graphicx}
\makeatletter
\def\maxwidth{\ifdim\Gin@nat@width>\linewidth\linewidth\else\Gin@nat@width\fi}
\def\maxheight{\ifdim\Gin@nat@height>\textheight\textheight\else\Gin@nat@height\fi}
\makeatother
% Scale images if necessary, so that they will not overflow the page
% margins by default, and it is still possible to overwrite the defaults
% using explicit options in \includegraphics[width, height, ...]{}
\setkeys{Gin}{width=\maxwidth,height=\maxheight,keepaspectratio}
% Set default figure placement to htbp
\makeatletter
\def\fps@figure{htbp}
\makeatother
\setlength{\emergencystretch}{3em} % prevent overfull lines
\providecommand{\tightlist}{%
  \setlength{\itemsep}{0pt}\setlength{\parskip}{0pt}}
\setcounter{secnumdepth}{-\maxdimen} % remove section numbering
\ifLuaTeX
\usepackage[bidi=basic]{babel}
\else
\usepackage[bidi=default]{babel}
\fi
\babelprovide[main,import]{brazilian}
% get rid of language-specific shorthands (see #6817):
\let\LanguageShortHands\languageshorthands
\def\languageshorthands#1{}
\ifLuaTeX
  \usepackage{selnolig}  % disable illegal ligatures
\fi
\IfFileExists{bookmark.sty}{\usepackage{bookmark}}{\usepackage{hyperref}}
\IfFileExists{xurl.sty}{\usepackage{xurl}}{} % add URL line breaks if available
\urlstyle{same} % disable monospaced font for URLs
\hypersetup{
  pdftitle={Bookdown Meus Resumos},
  pdfauthor={Daniel Claudino},
  pdflang={pt-BR},
  hidelinks,
  pdfcreator={LaTeX via pandoc}}

\title{Bookdown Meus Resumos}
\author{Daniel Claudino}
\date{2022-11-06}

\begin{document}
\frontmatter
\maketitle

\mainmatter
\hypertarget{apresentauxe7uxe3o}{%
\chapter{Apresentação}\label{apresentauxe7uxe3o}}

Bookdown Meus Resumos

\begin{figure}

{\centering \includegraphics[width=0.5\linewidth]{imagens/FOTO-PERFIL-DANIEL-CLAUDINO-2020} 

}

\caption{Autor: Daniel Claudino}\label{fig:unnamed-chunk-1}
\end{figure}

Neste bookdown estarão contidos os resumos de: * Capítulos de livros *
Artigos * Monografias * Dissertações * Teses * Notícias de jornais

Esses materiais estarão relacionados com as disciplinas do 1º até o 10º
período do curso de Bacharelado em Psicologia, bem como com a elaboração
do meu TCC, artigos, dissertações e teses a serem publicados por mim.

\hypertarget{controle-de-versuxe3o}{%
\section{Controle de Versão}\label{controle-de-versuxe3o}}

\begin{longtable}[]{@{}
  >{\raggedright\arraybackslash}p{(\columnwidth - 6\tabcolsep) * \real{0.2500}}
  >{\raggedright\arraybackslash}p{(\columnwidth - 6\tabcolsep) * \real{0.2500}}
  >{\raggedright\arraybackslash}p{(\columnwidth - 6\tabcolsep) * \real{0.2500}}
  >{\raggedright\arraybackslash}p{(\columnwidth - 6\tabcolsep) * \real{0.2500}}@{}}
\toprule()
\begin{minipage}[b]{\linewidth}\raggedright
Versão
\end{minipage} & \begin{minipage}[b]{\linewidth}\raggedright
Data / Hora
\end{minipage} & \begin{minipage}[b]{\linewidth}\raggedright
Colaborador
\end{minipage} & \begin{minipage}[b]{\linewidth}\raggedright
Descrição da Contribuição
\end{minipage} \\
\midrule()
\endhead
0.1 & 01/11/2022 11h34 & \href{https://wa.me/5583988853815}{Daniel
Claudino} & Versão inicial do documento \\
\bottomrule()
\end{longtable}

\hypertarget{referuxeancias-bibliogruxe1ficas}{%
\section{Referências
Bibliográficas}\label{referuxeancias-bibliogruxe1ficas}}

\hypertarget{bibliografia-buxe1sica}{%
\subsection{Bibliografia Básica}\label{bibliografia-buxe1sica}}

DAVIDOFF, Linda L. Introdução à Psicologia. São Paulo: Makron Books,
2001.

SPINK, M. J. P. Psicologia social e saúde: práticas, saberes e sentidos.
Petrópolis:Vozes, 2013.

MAISTO, Albert A.; MORRIS, Charles G. Introdução a Psicologia. 6 ed.~São
Paulo, Prentice Hall, 2004. {[}Livro Eletrônico{]}

\hypertarget{bibliografia-complementar}{%
\subsection{Bibliografia Complementar}\label{bibliografia-complementar}}

BRIGAGÃO, J., NASCIMENTO, V. L. V., \& SPINK, P. K. (2011). As
interfaces entre psicologia e políticas públicas e a configuração de
novos espaços de atuação. Sorocaba, (páginas, 199-215).

CASTRO, E. K., \& BORNHOLDT, E. (2004). Psicologia da saúde x psicologia
hospitalar: definições e possibilidades de inserção profissional.
Psicologia Ciência e Profissão (páginas, 48-57).

COELHO, Wilson Ferreira. Psicologia do Desenvolvimento. São Paulo:
Editora Pearson, 2014. {[}Livro Eletrônico{]}

CÓRIA-SABINI, Maria Aparecida. Psicologia do Desenvolvimento. 2 ed.~São
Paulo: Editora Ática, 2010. {[}Livro Eletrônico{]}

DIAS, A. C. G., PATIAS, N. D., \& ABAID, J. L. W. ( 2014). Psicologia
escolar e possibilidades na atuação do psicólogo: algumas reflexões.
Revista Psicologia Escolar e Educacional (páginas 105-111).

FEIST, J., FEIST, G., \& ROBERTS, T. A. (2015). Teorias da
Personalidade.

FELDMAN, Robert S. Introdução à Psicologia. Porto Alegre: Editora
AMGH,2015.

ILETTI, Nelson; ROSSATO, Solange Marques; ROSSATO, Geovanio. Psicologia
do Desenvolvimento. São Paulo, Contexto, 2014. {[}Livro Eletrônico{]}

LIMA, C. F., \& PIMENTEL, C. E. (2017). Livro: Revisitando a Psicologia
Social. MISKOLCI, Richard. Teoria Queer: um aprendizado pelas
diferenças. 2 ed.~Belo Horizonte: Autêntica, 2015. {[}Livro
Eletrônico{]}

PADILHA, S., NORONHA, A. P. P., \& ZANCHET, C. F. (2007). Instrumentos
de avaliação psicológica: uso e parecer de psicólogos. Avaliação
psicológica (páginas, 69-79).

SCHULTZ, D. \& SCHULTZ, S. E. (2019). História da Psicologia moderna.

ZANELLI, J. C., BASTOS, A. V. B., \& RODRIGUES, A. C. A. ( 2014).
Psicologia, Organizações e Trabalho no Brasil. (Orgs).

\hypertarget{observauxe7uxe3o-importante}{%
\section{Observação Importante}\label{observauxe7uxe3o-importante}}

\textbf{NOTA}: Este material tem como finalidade auxiliar a fixação de
assuntos estudados em sala de aula de acordo com o \textbf{plano de
ensino desta disciplina}.

Ele \textbf{não deve ser} utilizado como \textbf{único material de
estudo para a prova}, então:

\begin{enumerate}
\def\labelenumi{\arabic{enumi}.}
\tightlist
\item
  Consulte os \textbf{slides da professora} na plataforma FTM;\\
\item
  Faça \textbf{notas de aula} do que for tratado em sala de aula;\\
\item
  Consulte nossas \textbf{notas de aula};
\end{enumerate}

\begin{quote}
\textbf{Dúvidas}: Devem ser encaminhadas no grupo de whatsapp da
disciplina.
\end{quote}

\hypertarget{p1---anatomia-humana}{%
\chapter{P1 - Anatomia Humana}\label{p1---anatomia-humana}}

Neste capítulo estarão contidos os resumos relacionados com a disciplina
Anatomia Humana.

Em breve\ldots{}

\hypertarget{p1---introduuxe7uxe3o-uxe0-psicologia}{%
\chapter{P1 - Introdução à
Psicologia}\label{p1---introduuxe7uxe3o-uxe0-psicologia}}

Neste capítulo estarão contidos os resumos relacionados com a disciplina
Introdução à Psicologia.

\hypertarget{livro-psicologias-uma-introduuxe7uxe3o-ao-estudo-da-psicologia-boch-furtado-teixeira-2001}{%
\section{\texorpdfstring{Livro: \textbf{Psicologias Uma Introdução ao
Estudo da Psicologia} (BOCH; FURTADO; TEIXEIRA,
2001)}{Livro: Psicologias Uma Introdução ao Estudo da Psicologia (BOCH; FURTADO; TEIXEIRA, 2001)}}\label{livro-psicologias-uma-introduuxe7uxe3o-ao-estudo-da-psicologia-boch-furtado-teixeira-2001}}

\begin{figure}

{\centering \includegraphics[width=0.5\linewidth]{imagens/capa-livro-psicologias} 

}

\caption{Livro Psicologias: Uma Introdução ao Estudo da Psicologia. 13.ed. São Paulo: Saraiva, 2001}\label{fig:unnamed-chunk-2}
\end{figure}

\hypertarget{capuxedtulo-3---o-behaviorismo}{%
\subsection{Capítulo 3 - O
Behaviorismo}\label{capuxedtulo-3---o-behaviorismo}}

\hypertarget{o-estudo-do-comportamento}{%
\subsubsection{O Estudo do
Comportamento}\label{o-estudo-do-comportamento}}

\begin{itemize}
\tightlist
\item
  O termo \textbf{Behaviorismo} foi inaugurado pelo americano
  \textbf{John B. Watson} num artigo publicado em \textbf{1913}
  intítulado ``\textbf{Psicologia: como os behavioristas a vêem}''
\item
  Para denominar a tendência teórica foi utilizdo a expressão
  Behaviorismo, derivado da palavra em inglês \emph{behavior} que
  significa Comportamento
\item
  Essa tendência teórica também é denominada:

  \begin{itemize}
  \tightlist
  \item
    Comportamentalismo;
  \item
    Teoria Comportamental;
  \item
    Análise Experimental do Comportamento;
  \item
    Análise do Comportamento
  \end{itemize}
\item
  \textbf{John B. Watson} deu a Psicologia a \textbf{consistência de
  ciência} que os psicólogos da época vinham procurando.

  \begin{itemize}
  \tightlist
  \item
    ``\textbf{Consistência de ciência}'' significa possuir um objeto com
    as características de ser mensurável, observável, cujos experimentos
    poderiam ser reproduzidos em diferentes condições e sujeitos
    rompendo definitivamente com a sua \emph{tradição filosófica}
  \end{itemize}
\item
  \textbf{John B. Watson} postulou \textbf{O Comportamento} como objeto
  da Psicologia
\item
  \textbf{John B. Watson} defendia uma \textbf{perspectiva
  funcionalista} para a Psicologia, significando que:

  \begin{itemize}
  \tightlist
  \item
    O comportamento deveria ser estudado cem função de certas variáveis
    do meio;
  \item
    Certos estímulos levam o organismo a dar determinadas respostas;
  \item
    Isso ocorre porque os organismos se ajustam aos seus ambientes:

    \begin{itemize}
    \tightlist
    \item
      Por meio de equipamentos hereditários;
    \item
      Pela formação de hábitos;
    \end{itemize}
  \end{itemize}
\item
  \textbf{John B. Watson} buscava a construção de uma Psicologia:

  \begin{itemize}
  \tightlist
  \item
    Sem alma;
  \item
    Sem mente;
  \item
    Livre de conceitos mentalistas;
  \item
    Livre de métodos subjetivos;
  \item
    Que tivesse a capacidade de \textbf{Prever} e \textbf{Controlar};
  \end{itemize}
\item
  Desde o início (1913), os behavioristas foram modificando o
  significado do termo ``\emph{comportamento}''

  \begin{itemize}
  \tightlist
  \item
    No início, o comportamento era visto como \textbf{ação isolada} do
    sujeito
  \item
    Posteriormentem, o comportamento passou a ser visto como uma
    interação entre aquilo que o sujeito faz e o ambiente onde o seu
    ``fazer'' acontece;
  \end{itemize}
\item
  Behaviorismo dedica-se ao estudo:

  \begin{itemize}
  \tightlist
  \item
    Das interações entre o indivíduo e o ambiente
  \item
    Entre as ações do indivíduo (suas respostas) e o ambiente (as
    estimulações)
  \end{itemize}
\item
  Os psicólogos desta abordagem chegaram ao seguinte entendimento

  \begin{itemize}
  \tightlist
  \item
    \textbf{Estímulo}: As variáveis ambientais queinteragem com o
    sujeito
  \item
    \textbf{Resposta}: Aquilo que o organismo faz
  \end{itemize}
\end{itemize}

\begin{itemize}
\tightlist
\item
  As \textbf{razões} para explicar a adoção dos termos \textbf{Estímulo}
  e \textbf{Resposta}

  \begin{itemize}
  \tightlist
  \item
    \textbf{Razão Metodológica}: Adoção de método analítico e
    experimental como modo preferencial de investigação;
  \item
    \textbf{Razão Histórica}: Devido ao seu \textbf{uso generalizado}.
    Os termos escolhidos eram popularizados que foram mantidos ao longo
    do tempo
  \end{itemize}
\item
  O \textbf{COMPORTAMENTO}

  \begin{itemize}
  \tightlist
  \item
    É a unidade básica de descrição;
  \item
    É o ponto de partida para uma ciência do comportamento
  \end{itemize}
\item
  O homem começa a ser

  \begin{itemize}
  \tightlist
  \item
    Estudado a partir de sua interação com o ambiente;
  \item
    Ser tomado como \textbf{produto} e \textbf{produtor} dessas
    interações.
  \end{itemize}
\end{itemize}

\hypertarget{a-anuxe1lise-experimental-do-comportamento}{%
\subsubsection{A Análise Experimental do
Comportamento}\label{a-anuxe1lise-experimental-do-comportamento}}

\begin{itemize}
\tightlist
\item
  Um dos mais importantes sucessores de \textbf{John B. Watson} foi
  \textbf{B. F. Skinner (1904-1990)}
\item
  O Behaviorismo de Skinner tem influenciado muitos psicólogos
  americanos e de vários países onde a Psicologia americana tem grande
  penetração, como o Brasil
\item
  Esta linha de estudo ficou conhecida por \textbf{Behaviorismo radical}
  (Skinner, em 1945)
\item
  A expressão \textbf{Behaviorismo radical} foi designada {[}por
  Skinner{]} como \textbf{uma filosofia da Ciência do Comportamento} por
  meio da \textbf{análise experimental do comportamento}.
\item
  O \textbf{comportamento operante} é a \textbf{BASE DO BEHAVIORISMO
  RADICAL} de Skinner
\item
  Para entender o \textbf{comportamento operante} é necessário
  compreender antes dois outros conceitos:

  \begin{itemize}
  \tightlist
  \item
    Comportamento Reflexo
  \item
    Comportamento Respondente
  \end{itemize}
\end{itemize}

\hypertarget{comportamento-respondente}{%
\paragraph{Comportamento Respondente}\label{comportamento-respondente}}

\begin{itemize}
\tightlist
\item
  O \textbf{COMPORTAMENTO REFLEXO} OU \textbf{COMPORTAMENTO RESPONDENTE}

  \begin{itemize}
  \tightlist
  \item
    É o que usualmente chamamos de ``não-voluntário'' e inclui as
    respostas que são eliciadas (``produzidas'') por estímulos
    antecedentes do ambiente.

    \begin{itemize}
    \tightlist
    \item
      Exemplo:

      \begin{enumerate}
      \def\labelenumi{\alph{enumi}.}
      \tightlist
      \item
        Contração das pupilas quando uma luz forte incide sobre os olhos
      \item
        Salivação provocada por uma gota de limão colocada na ponta da
        língua
      \item
        Arrepio da pele quando um ar frio nos atinge
      \item
        Lacrimejar ao cortar cebola;
      \end{enumerate}
    \end{itemize}
  \item
    São interações estímulo-resposta (ambiente-sujeito)

    \begin{itemize}
    \tightlist
    \item
      \textbf{INCONDICIONADAS}
    \item
      Nos quais certos eventos ambientais \textbf{confiavelmente}
      eliciam certas respostas do organismo que INDEPENDEM de
      ``aprendizagem''
    \item
      Que também podem ser provocadas por estímulos que, originalmente,
      não eliciavam respostas em determinado organismo
    \end{itemize}
  \end{itemize}
\item
  Quando \textbf{ESTÍMULOS QUE ORIGINALMENTE NÃO ELICIAVAM CERTAS
  RESPOSTAS} são pareados com estímulos eliciadores, em certas
  condições, podem elicitar respostas semelhantes às destes.
\item
  As \textbf{NOVAS INTERAÇÕES (estímulo-resposta)} são também chamados
  de \textbf{COMPORTAMENTO REFLEXO} ( INTERAÇÕES CONDICIONADAS devido a
  história de pareamento o qual levou o organismo a responder a
  estímulos que antes não respondia ).

  \begin{itemize}
  \tightlist
  \item
    Exemplo:

    \begin{enumerate}
    \def\labelenumi{\arabic{enumi}.}
    \tightlist
    \item
      Suponha que, numa sala aquecida, sua mão direita seja mergulhada
      numa vasilha de água gelada
    \item
      A temperatura da mão cairá rapidamente devido ao encolhimento ou
      constrição dos vasos sangüíneos, caracterizando o comportamento
      como respondente
    \item
      Esse comportamento será acompanhado de uma modificação semelhante,
      e mais facilmente mensurável, na mão esquerda, onde a constrição
      vascular também será induzida.
    \item
      Suponha, agora, que a sua mão direita seja mergulhada na água
      gelada um certo número de vezes, em intervalos de três ou quatro
      minutos, e que você ouça uma campainha pouco antes de cada
      imersão.
    \item
      Lá pelo vigésimo pareamento do som da campainha com a água fria, a
      mudança de temperatura nas mãos poderá ser eliciada apenas pelo
      som, isto é, sem necessidade de imergir uma das mãos
    \end{enumerate}

    \begin{itemize}
    \tightlist
    \item
      Análise:

      \begin{enumerate}
      \def\labelenumi{\alph{enumi}.}
      \tightlist
      \item
        A queda da temperatura da mão, eliciada pela água fria, é uma
        resposta incondicionada
      \item
        A queda da temperatura, eliciada pelo som, é uma resposta
        condicionada (aprendida)
      \item
        A água é um estímulo incondicionado, e o som, um estímulo
        condicionado
      \end{enumerate}
    \end{itemize}
  \end{itemize}
\item
  No início dos anos 30, na Universidade de Harvard (Estados Unidos),
  Skinner começou o estudo do comportamento justamente pelo
  comportamento respondente

  \begin{itemize}
  \tightlist
  \item
    O \textbf{COMPORTAMENTO RESPONDENTE} que se tornou:

    \begin{itemize}
    \tightlist
    \item
      A unidade básica de análise, ou seja, o fundamento para a
      descrição das interações indivíduo ambiente.
    \end{itemize}
  \item
    O desenvolvimento de seu trabalho levou-o a teorizar sobre um outro
    tipo de relação do indivíduo com seu ambiente, a qual viria a ser
    \textbf{NOVA UNIDADE DE ANÁLISE} de sua ciência: \textbf{o
    comportamento}
  \item
    Esse \textbf{tipo de comportamento} caracteriza a \textbf{maioria de
    nossas interações com o ambiente}.
  \end{itemize}
\end{itemize}

\hypertarget{o-comportamento-operante}{%
\paragraph{O Comportamento Operante}\label{o-comportamento-operante}}

EXPERIMENTO DA CAIXA DE SKINNER

\begin{figure}

{\centering \includegraphics[width=0.9\linewidth]{imagens/p1-caixa-de-skinner} 

}

\caption{Caixa de Skinner}\label{fig:unnamed-chunk-4}
\end{figure}

\begin{enumerate}
\def\labelenumi{\arabic{enumi}.}
\tightlist
\item
  Um ratinho colocado na ``caixa de Skinner'' um recipiente fechado no
  qual encontrava apenas uma barra.
\item
  Esta barra, ao ser pressionada por ele, acionava um mecanismo
  (camuflado) que lhe permitia uma gotinha de água, que chegava à caixa
  por meio de uma pequena haste
\item
  Que resposta esperava-se do ratinho? --- Que pressionasse a barra.
\item
  Como isso ocorreu pela primeira vez? --- Por acaso. Durante a
  exploração da caixa, o ratinho pressionou a barra acidentalmente, o
  que lhe trouxe, pela primeira vez, uma gotinha de água, que, devido à
  sede, fora rapidamente consumida.
\item
  Por ter obtido água ao encostar na barra quando sentia sede,
  constatou-se a alta probabilidade de que, \textbf{estando em situação
  semelhante}, o ratinho a pressionasse novamente.
\end{enumerate}

\begin{itemize}
\item
  Inclui todos os movimentos de um organismo dos quais se possa dizer
  que, em algum momento, têm efeito sobre ou fazem algo ao mundo em
  redor.
\item
  O comportamento operante opera sobre o mundo, por assim dizer, quer
  direta, quer indiretamente
\item
  Desde o início do Behaviorismo, animais (pombos, ratos e macacos)
  foram usados para verificar como as \textbf{variações no ambiente}
  interferiam nos \textbf{comportamentos}.

  \begin{itemize}
  \tightlist
  \item
    Um ratinho, ao sentir sede em seu habitat, certamente manifesta
    algum comportamento que lhe permita satisfazer a sua necessidade
    orgânica
  \item
    Se deixarmos um ratinho privado de água durante 24 horas, ele
    certamente apresentará o comportamento de beber água no momento em
    que tiver sede
  \end{itemize}
\item
  Comportamento Operante:

  \begin{itemize}
  \tightlist
  \item
    Comportamento foi aprendido e que se mantém pelo efeito
    proporcionado: \textbf{saciar a sede}
  \end{itemize}
\item
  Os pesquisadores da época decidiram \textbf{simular esta situação em
  laboratório}, SOB CONDIÇÕES ESPECIAIS DE CONTROLE, o que os levou à
  formulação de uma \textbf{LEI COMPORTAMENTAL}
\item
  Neste caso de COMPORTAMENTO OPERANTE:

  \begin{itemize}
  \tightlist
  \item
    O que propicia a aprendizagem dos comportamentos ?

    \begin{itemize}
    \tightlist
    \item
      É a ação do organismo sobre o meio
    \item
      É o efeito dela resultante --- a satisfação de alguma necessidade,
      ou seja, a aprendizagem está na relação entre \textbf{UMA AÇÃO} e
      seu \textbf{EFEITO}.
    \end{itemize}
  \end{itemize}
\item
  Este comportamento operante pode ser representado da seguinte maneira:
  R ---► S, em que R é a resposta (pressionar a barra) e S (doinglês
  stimuli) o estímulo reforçador (a água), que tanto interessa ao
  organismo; a flecha significa ``levar a''.
\item
  Esse estímulo reforçador é chamado de reforço.
\item
  O termo ``estímulo'' foi mantido da relação R-S do comportamento
  respondente para designar-lhe a responsabilidade pela ação, apesar de
  ela ocorrer após a manifestação do comportamento.
\item
  O \textbf{COMPORTAMENTO OPERANTE} refere-se à interação
  sujeito-ambiente. Nessa interação:

  \begin{itemize}
  \tightlist
  \item
    Chama-se de relação fundamental à relação entre a ação do indivíduo
    (a emissão da resposta) e as conseqüências.
  \item
    É considerada fundamental porque o organismo se comporta (emitindo
    esta ou aquela resposta), sua ação produz uma alteração ambiental
    (uma conseqüência) que, por sua vez, retroage sobre o sujeito,
    alterando a probabilidade futura de ocorrência.
  \end{itemize}
\item
  Agimos ou operamos sobre o mundo em função das conseqüências criadas
  pela nossa ação.
\item
  As conseqüências da resposta são as variáveis de controle mais
  relevantes.
\item
  Exemplo:

  \begin{itemize}
  \tightlist
  \item
    Pense no aprendizado de um instrumento: nós o tocamos para ouvir seu
    som harmonioso.
  \item
    Há outros exemplos: podemos dançar para estar próximo do corpo do
    outro, mexer com uma garota para receber seu olhar, abrir uma janela
    para entrar a luz etc.
  \end{itemize}
\end{itemize}

\hypertarget{capuxedtulo-4---a-gestalt}{%
\subsection{Capítulo 4 - A Gestalt}\label{capuxedtulo-4---a-gestalt}}

\hypertarget{a-psicologia-da-forma-introduuxe7uxe3o-uxe0-psicologia-da-gestalt}{%
\subsubsection{A Psicologia da Forma: Introdução à Psicologia da
Gestalt}\label{a-psicologia-da-forma-introduuxe7uxe3o-uxe0-psicologia-da-gestalt}}

\begin{itemize}
\tightlist
\item
  Para Bock (2001, p.~59) a Psicologia da Gestalt é uma das tendências
  teóricas mais \textbf{coerentes} e \textbf{coesas} da história da
  psicologia.
\item
  O termo Gestalt é de origem alemã e tem significado aproximado ao de
  \textbf{forma} ou \textbf{configuração}, \textbf{porém} \textbf{NÃO É
  UTILIZADO} por não corresponder exatamente as seu real significado em
  psicologia.
\item
  No final do século XIII, estudiosos procuravam compreender o
  \textbf{fenômeno psicológico} em seus aspectos naturais.

  \begin{itemize}
  \tightlist
  \item
    Principalmente no sentido da \textbf{mensurabilidade} ( A
    Psicofísica em voga ).
  \end{itemize}
\end{itemize}

\hypertarget{predecessores-da-psiologia-da-gestalt}{%
\paragraph{Predecessores da Psiologia da
Gestalt}\label{predecessores-da-psiologia-da-gestalt}}

\begin{itemize}
\tightlist
\item
  Estudiosos considerados os mais diretos predecessores/antecessores da
  Psisocologia Gestalt:

  \begin{itemize}
  \tightlist
  \item
    Ernst Mash (1838-1916), físico;
  \item
    Christian von Ehrenfels (1859-1932), fisólofo e psicólogo
  \end{itemize}
\item
  Estudos desenvolvidos:

  \begin{itemize}
  \tightlist
  \item
    Estudos psicofísicos sobre as \textbf{sensações} de
    \textbf{espaço-forma} e \textbf{tempo-forma}

    \begin{itemize}
    \tightlist
    \item
      Dado Psicológico: Sensações
    \item
      Dados Físico: espaço-forma e tempo-forma
    \end{itemize}
  \end{itemize}
\item
  Fundadores

  \begin{itemize}
  \tightlist
  \item
    Max Wertheimer
  \item
    Wolfgang Kohler
  \item
    Kurt Koffka
  \end{itemize}
\end{itemize}

\hypertarget{fundadores-da-psiologia-da-gestalt}{%
\paragraph{Fundadores da Psiologia da
Gestalt}\label{fundadores-da-psiologia-da-gestalt}}

\begin{itemize}
\tightlist
\item
  Os fundadores da Psicologia da Gestalt construíram a \textbf{base de
  uma teoria psicológica}.
\end{itemize}

\begin{longtable}[]{@{}
  >{\centering\arraybackslash}p{(\columnwidth - 4\tabcolsep) * \real{0.3333}}
  >{\centering\arraybackslash}p{(\columnwidth - 4\tabcolsep) * \real{0.3333}}
  >{\centering\arraybackslash}p{(\columnwidth - 4\tabcolsep) * \real{0.3333}}@{}}
\toprule()
\endhead
Figura -
\href{https://translate.google.com/?sl=hu\&tl=pt\&text=Max\%20Wertheimer\&op=translate}{Max
Wertheimer :speaker:} (1880-9343) & Figura -
\href{https://translate.google.com/?sl=de\&tl=pt\&text=Wolfgang\%20Kohler\&op=translate}{Wolfgang
Kohler :speaker:}(1887-1967) & Figura -
\href{https://translate.google.com/?sl=de\&tl=pt\&text=Kurt\%20Koffka\&op=translate}{Kurt
Koffka :speaker:}(1886-1941) \\
\bottomrule()
\end{longtable}

\textbf{Obs}: Clique em :speaker: para ouvir a pronúncia dos nomes dos
cientístas acima.

\begin{itemize}
\tightlist
\item
  Estudos iniciais

  \begin{itemize}
  \tightlist
  \item
    Estudos da percepção e sensação do movimento;
  \item
    Preocupação: Compreender \textbf{quais processos psicológicos}
    estavam envolvidos na \textbf{ilusão de ótica} quando o estímulo é
    percebido como uma \textbf{forma} diferente da que o sujeito tem na
    realidade.

    \begin{itemize}
    \tightlist
    \item
      Exemplo: Cinema; fotogramas estáticos; imagem formada na retina
      que demora um pouco para ser apagada; ilusão de óptica do
      movimento (sensação).
    \end{itemize}
  \end{itemize}
\end{itemize}

\hypertarget{a-percepuxe7uxe3o}{%
\subsubsection{A percepção}\label{a-percepuxe7uxe3o}}

\begin{itemize}
\tightlist
\item
  É \textbf{ponto de partida} e \textbf{tema central} da Psicologia da
  Gestalt;
\item
  Teoria Behaviorista

  \begin{itemize}
  \tightlist
  \item
    \textbf{Princípio Implícito}: Há uma relação de \textbf{causa} e
    \textbf{efeito} entre o \textbf{estímulo} e a \textbf{resposta}
  \end{itemize}
\item
  Para Gestaltistas há um questionamento desse \textbf{princípio
  implícito da teoria behaviorista}

  \begin{itemize}
  \tightlist
  \item
    Entre o \textbf{estímulo} e a \textbf{resposta} encontra-se o
    ****processo de percepção****
  \item
    ****O QUE**** o indivíduo percebe e ****COMO**** o indivíduo percebe
    ****são importantes para entender o COMPORTAMENTO****
  \end{itemize}
\end{itemize}

\begin{Shaded}
\begin{Highlighting}[]
\NormalTok{flowchart LR}
\NormalTok{OQ(O que){-}{-}A Pessoa Percebe{-}{-}\textgreater{}CC(Entendimento do Comportamento)}
\NormalTok{CM(Como){-}{-}A Pessoa Percebe{-}{-}\textgreater{}CC}
\end{Highlighting}
\end{Shaded}

\hypertarget{posiuxe7uxe3o-de-behavioristas-x-gestaltistas-diante-do-objeto-da-psicologia}{%
\paragraph{Posição de Behavioristas x Gestaltistas diante do Objeto da
Psicologia}\label{posiuxe7uxe3o-de-behavioristas-x-gestaltistas-diante-do-objeto-da-psicologia}}

\begin{itemize}
\tightlist
\item
  Ambos definem a psicologia como a ****ciência que estuda o
  COMPORTAMENTO****
\item
  Para os Behavioristas:

  \begin{itemize}
  \tightlist
  \item
    É mais profunda a preocupação com a \textbf{objetividade};
  \item
    O \textbf{estudo com comportamento} é feito através da
    \textbf{relação estímulo-resposta};
  \item
    Despreza os ****conteúdos da consciência**** pela impossibilidade de
    controlar cientificamente \textbf{essas variáveis};
  \item
    Procura isolar o \textbf{estímulo} que corresponderia à
    \textbf{resposta} desprezando ****conteúdos da consciência**** pela
    impossibilidade de controlar cientificamente \textbf{essas
    variáveis};
  \end{itemize}
\item
  Para os gestaltistas:

  \begin{itemize}
  \tightlist
  \item
    Há uma \textbf{crítica} a \textbf{abordagem behaviorista} acima;
  \item
    Acreditam que existe um \textbf{contexto mais amplo} que é
    importante no \textbf{estudo do comportamento}

    \begin{itemize}
    \tightlist
    \item
      Esse \textbf{contexto mais amplo} são as ****CONDIÇÕES**** que
      \textbf{afetam/alteram} nossa capacidade de ****PERCEBER**** o
      \textbf{estímulo};
    \end{itemize}
  \item
    Entendem que estudar o comportamento isolado de um \textbf{contexto
    mais amplo} pode prejudicar o entendimento do comportamento pelo
    psicólogo;
  \item
    O comportamento é estudado em seus aspectos mais globais levando em
    consideração as ****CONDIÇÕES**** que \textbf{afetam/alteram} nossa
    capacidade de ****PERCEBER**** o \textbf{estímulo}
  \end{itemize}
\end{itemize}

\hypertarget{o-que-garante-o-entendimento-do-que-eu-percebo}{%
\paragraph{O que Garante o Entendimento do que Eu Percebo
?}\label{o-que-garante-o-entendimento-do-que-eu-percebo}}

\begin{itemize}
\tightlist
\item
  Quando eu vejo

  \begin{itemize}
  \tightlist
  \item
    Uma parte de um objeto
  \end{itemize}
\item
  Ocorre uma tendência à

  \begin{itemize}
  \tightlist
  \item
    restauração do \textbf{equilíbrio da forma}
  \end{itemize}
\item
  Garantindo * O entendimento do que estou percebendo
\end{itemize}

\hypertarget{o-fenuxf4meno-da-percepuxe7uxe3o}{%
\paragraph{O Fenômeno da
Percepção}\label{o-fenuxf4meno-da-percepuxe7uxe3o}}

\begin{itemize}
\tightlist
\item
  É norteado pela busca de

  \begin{itemize}
  \tightlist
  \item
    \textbf{fechamento} dos pontos que compõem uma figura;
  \item
    \textbf{simetria} dos pontos que compõem uma figura;
  \item
    \textbf{regularidade} dos pontos que compõem uma figura;
  \end{itemize}
\item
  \textbf{Rudolf Arnheim} afirma que o \textbf{sentido normal da visão}
  apreende um \textbf{padrão global};
\end{itemize}

Figura - Lei básica da percepção visual para os psicólogos da Gestalt

\begin{itemize}
\tightlist
\item
  Observações a respeito da Figura:

  \begin{itemize}
  \tightlist
  \item
    \textbf{Figura 1}:

    \begin{itemize}
    \tightlist
    \item
      Percebemos como um \textbf{quadrado} e não como uma \textbf{figura
      inclinada} ou um \textbf{perfil} (Figura 2)
    \end{itemize}
  \item
    \textbf{Figura 3}:

    \begin{itemize}
    \tightlist
    \item
      Após acrescentarmos quatro pontos, o padrão percebido na Figura 1
      irá mudar e perceberemos \textbf{um círculo}
    \end{itemize}
  \item
    \textbf{Figura 4}:

    \begin{itemize}
    \tightlist
    \item
      É possível ver tanto \textbf{círculos brancos} quanto
      \textbf{quadrados brancos} nos centros das cruzes;
    \end{itemize}
  \end{itemize}
\end{itemize}

\begin{quote}
Qualquer padrão de estímulo tende a ser visto de tal modo que a
\textbf{estrutura resultante} é tão \textbf{simples} quanto as
\textbf{condições dadas} permitem
\end{quote}

\hypertarget{a-boa-forma}{%
\subsubsection{A boa-forma}\label{a-boa-forma}}

\begin{itemize}
\tightlist
\item
  A Psicologia da Gestalt encontra as \textbf{condições} para a
  \textbf{compreensão do comportamento humano} nos \textbf{fenômenos da
  percepção}.
\item
  Em relação aos nossos comportamentos:

  \begin{itemize}
  \tightlist
  \item
    Em alguns casos, \textbf{guardam estreita relação com os estímulos
    físicos};
  \item
    Em outros casos, \textbf{são completamente diferentes do esperado}
    porque ``****entendemos o ambiente****'' \textbf{de maneira
    diferente da sua realidade}.
  \item
    Exemplo:

    \begin{itemize}
    \tightlist
    \item
      Cumprimentar uma pessoas e depois descobrir que cumprimentamos uma
      pessoa desconhecida (\textbf{Erro de Percepção});
    \end{itemize}
  \end{itemize}
\item
  Não há \textbf{boa forma} quando \textbf{nos elementos percebidos} não
  há:

  \begin{itemize}
  \tightlist
  \item
    Equilíbrio
  \item
    Simetria
  \item
    Estabilidade
  \item
    Simplicidade
  \end{itemize}
\item
  \textbf{O elemento que objetivamos compreender}

  \begin{itemize}
  \tightlist
  \item
    COMO DEVE SER APRESENTADO

    \begin{itemize}
    \tightlist
    \item
      Deve ser apresentado em \textbf{aspectos básicos} que
      \textbf{permitam a sua decodificação} (\textbf{percepção da boa
      forma})
    \end{itemize}
  \end{itemize}
\end{itemize}

Figura - Observe a RETA: Elemento que desejamos compreender

\begin{itemize}
\tightlist
\item
  (\ldots)

  \begin{itemize}
  \tightlist
  \item
    COMO DEVEM SER DISTRIBUÍDOS OS ELEMENTOS QUE O COMPÕEM * Para
    garantir a \textbf{boa forma} devem ser apresentados com -
    Equilíbrio - Simetria - Estabilidade - Simplicidade
  \end{itemize}
\item
  A \textbf{tendência da nossa percepção} em buscar a \textbf{boa forma}
  permitirá a relação \textbf{figura-fundo}
\item
  Quanto mais clara (simples, estável, simétrica e equilibrada) estiver
  a \textbf{boa-forma}

  \begin{itemize}
  \tightlist
  \item
    Mais clara será a \textbf{separação} entre \textbf{figura} e
    \textbf{fundo};
  \end{itemize}
\item
  Quanto menos clara estiver a boa forma

  \begin{itemize}
  \tightlist
  \item
    Mais difícil será distinguir o que é figura e o que é fundo
    (****Figura Ambígua****);
  \end{itemize}
\end{itemize}

\hypertarget{meio-geogruxe1fico-e-meio-comportamental}{%
\subsubsection{Meio Geográfico e Meio
Comportamental}\label{meio-geogruxe1fico-e-meio-comportamental}}

\begin{itemize}
\tightlist
\item
  O comportamento \textbf{É DETERMINADO} pela \textbf{PERCEPÇÃO DO
  ESTÍMULO};
\item
  O comportamento está/estará sujeito a \textbf{LEI DA BOA-FORMA};
\item
  O \textbf{CONJUNTO DE ESTÍMULOS} determinantes do comportamento é
  denominado \textbf{MEIO AMBIENTAL} ( ou apenas MEIO )
\item
  Existem DOIS TIPOS DE MEIOS AMBIENTAIS

  \begin{itemize}
  \tightlist
  \item
    \textbf{Meio GEOGRÁFICO}

    \begin{itemize}
    \tightlist
    \item
      É o meio enquanto tal;
    \item
      É o meio físico EM TERMOS OBJETIVOS;
    \end{itemize}
  \item
    \textbf{Meio COMPORTAMENTAL}

    \begin{itemize}
    \tightlist
    \item
      É o meio resultante de INTERAÇÃO (Indivíduo
      \textless==\textgreater{} Meio Físico)
    \item
      Implica a \textbf{INTERPRETAÇÃO} desse meio através das
      \textbf{FORÇAS} que regem a \textbf{PERCEPÇÃO};

      \begin{enumerate}
      \def\labelenumi{\alph{enumi}.}
      \tightlist
      \item
        Forças que regem a percepção:
      \item
        Equilíbrio
      \end{enumerate}

      \begin{enumerate}
      \def\labelenumi{\roman{enumi}.}
      \setcounter{enumi}{1}
      \tightlist
      \item
        Simetria
      \item
        Estabilidade
      \item
        Simplicidade
      \end{enumerate}
    \end{itemize}
  \end{itemize}
\item
  Exemplo

  \begin{itemize}
  \tightlist
  \item
    Cumprimentar uma pessoa desconhecida
  \item
    Se só tivéssemos o \textbf{MEIO GEOGRÁFICO}, essa seria a nossa
    ÚNICA POSSIBILIDADE de percepção;
  \item
    A \textbf{SITUAÇÃO} levou-nos a uma \textbf{INTERPRETAÇÃO DIFERENTE
    DA REALIDADE} e ocorre a confusão com uma pessoa conhecida

    \begin{itemize}
    \tightlist
    \item
      DADOS DA SITUAÇÃO:

      \begin{enumerate}
      \def\labelenumi{\alph{enumi}.}
      \tightlist
      \item
        Encontro casual
      \item
        Encontro em movimento
      \item
        Impulso em manifestar uma reação ao encontro
      \end{enumerate}
    \end{itemize}
  \item
    No caso desse exemplo

    \begin{itemize}
    \tightlist
    \item
      A semelhança entre as duas pessoas foi \textbf{A CAUSA DO
      ENGANO(=COMPORTAMENTO)}
    \item
      Houve uma tendência em ESTABELECER A UNIDADE DE SEMELHANÇAS entre
      as duas pessoas, MAIS QUE SUAS DIFERÊNÇAS.
    \end{itemize}
  \end{itemize}
\item
  Essa \textbf{TENDÊNCIA A ``JUNTAR'' OS ELEMENTOS} é o que a Gestalt
  denomina de \textbf{FORÇA DE CAMPO PSICOLÓGICO;}
\item
  Nessa \textbf{PARTICULAR INTERPRETAÇÃO DO MEIO} (= O MEIO AMBIENTAL)

  \begin{itemize}
  \tightlist
  \item
    O que PERCEBEMOS é ``UMA REALIDADE'':

    \begin{itemize}
    \tightlist
    \item
      Realidade \textbf{PARTICULAR}
    \item
      Realidade \textbf{OBJETIVA}
    \item
      Realidade \textbf{CRIADA POR NOSSA MENTE}
    \end{itemize}
  \end{itemize}
\end{itemize}

\hypertarget{campo-psicoluxf3gico}{%
\subsubsection{Campo Psicológico}\label{campo-psicoluxf3gico}}

\begin{itemize}
\tightlist
\item
  Campo psicológico é uma tendência que \textbf{garante} (1) a busca
  pela melhor forma possível \textbf{em situações que não estão muito
  claras}.
\end{itemize}

\begin{Shaded}
\begin{Highlighting}[]
\NormalTok{flowchart LR}
\NormalTok{A(\textless{}b\textgreater{}CAMPO\textless{}br\textgreater{}PSICOLÓGICO\textless{}/b\textgreater{}\textless{}br\textgreater{}Um Processo){-}{-}\textgreater{}B(Uma \textless{}/b\textgreater{}TENDÊNCIA\textless{}/b\textgreater{})}
\NormalTok{B{-}{-}Garante{-}{-}\textgreater{}C(Busca da MELHOR FORMA\textless{}br\textgreater{}possível)}
\NormalTok{B{-}{-}Que Ocorre\textless{}br\textgreater{}Quando nos\textless{}br\textgreater{}Deparamos{-}{-}\textgreater{}D(Com \textless{}/b\textgreater{}SITUAÇÕES NÃO\textless{}br\textgreater{}MUITO CLARAS\textless{}/b\textgreater{})}
\end{Highlighting}
\end{Shaded}

\hypertarget{princuxedpios-do-campo-psicoluxf3gico}{%
\paragraph{Princípios do Campo
Psicológico}\label{princuxedpios-do-campo-psicoluxf3gico}}

\begin{itemize}
\tightlist
\item
  O campo psicológico é um processo que ocorre de acordo com
  \textbf{PRINCÍPIOS}:
\end{itemize}

\begin{Shaded}
\begin{Highlighting}[]
\NormalTok{flowchart LR}
\NormalTok{A(\textless{}b\textgreater{}CAMPO PSICOLÓGICO\textless{}/b\textgreater{}){-}{-}Possui{-}{-}\textgreater{}B(PRINCÍPIOS)}
\NormalTok{B{-}{-}\textgreater{}C(Proximidade){-}{-}\textgreater{}F(Os ELEMENTOS mais próximos\textless{}br\textgreater{}tendem a ser AGRUPADOS)}
\NormalTok{B{-}{-}\textgreater{}D(Semelhança){-}{-}\textgreater{}G(Os ELEMENTOS SEMELHANTES\textless{}br\textgreater{}são AGRUPADOS)}
\NormalTok{B{-}{-}\textgreater{}E(Fechamento){-}{-}\textgreater{}H(Ocorre uma TENDÊNCIA\textless{}br\textgreater{}de COMPLETAR os\textless{}br\textgreater{}ELEMENTOS FALTANTES)}
\end{Highlighting}
\end{Shaded}

\hypertarget{insight}{%
\subsubsection{Insight}\label{insight}}

\begin{itemize}
\tightlist
\item
  Significa \textbf{COMPREENSÃO IMEDIATA};
\item
  Existe uma diferença em como duas correntes psicológicas concebem o
  *\textbf{processo de aprendizagem}:

  \begin{itemize}
  \tightlist
  \item
    \textbf{A Gestalt}

    \begin{itemize}
    \tightlist
    \item
      Acredita que a APRENDIZAGEM é uma RELAÇÃO entre o TODO e a PARTE;
    \end{itemize}
  \item
    \textbf{O Associativismo / Behaviorismo}

    \begin{itemize}
    \tightlist
    \item
      Acredita que a APRENDIZAGEM é uma relação de coisas MAIS SIMPLES
      para coisas MAIS COMPLEXAS;
    \end{itemize}
  \end{itemize}
\item
  Na perspectiva da Geltalt, a APRENDIZAGEM é uma relação entre o TODO e
  a PARTE

  \begin{itemize}
  \tightlist
  \item
    Exemplo: É possível uma criança de 03 anos, que não sabe ler,
    distinguir a marca de um refrigerante e nomeá-lo corretamente.

    \begin{itemize}
    \tightlist
    \item
      Ela identificou e SEPAROU a PALAVA em sua TOTALIDADE, distinguindo
      a PALAVRA(figura) e o FUNDO;
    \item
      A criança aprendeu a ler a PALAVRA não juntando as letras, mas
      DANDO SIGNIFICADO ao TODO;
    \end{itemize}
  \end{itemize}
\item
  Nem sempre \textbf{AS SITUAÇÕES VIVIDAS} se apresentam DE FORMA CLARA
  de maneira a permitir uma PERCEPÇÃO IMEDIATA.

  \begin{itemize}
  \tightlist
  \item
    Essas situações DIFICULTAM O PROCESSO DE APRENDIZADO, porque não
    permitem uma clara definição da FIGURA-FUNDO, impedindo a relação
    PARTE/TODO
  \end{itemize}
\end{itemize}

\hypertarget{explicauxe7uxe3o-do-fenuxf4meno-de-insight}{%
\paragraph{Explicação do Fenômeno de
INSIGHT}\label{explicauxe7uxe3o-do-fenuxf4meno-de-insight}}

\begin{itemize}
\tightlist
\item
  Às vezes, estamos olhando para um \textbf{FIGURA} que \textbf{não tem
  sentido para nós}
\item
  De repente, sem que tenhamos feito nenhum esforço especial, \textbf{A
  RELAÇÃO FIGURA-FUNDOSE ESTABELECE}.
\end{itemize}

\hypertarget{teoria-de-campo-de-kurt-lewin}{%
\subsubsection{\texorpdfstring{Teoria de Campo de \textbf{Kurt
Lewin}}{Teoria de Campo de Kurt Lewin}}\label{teoria-de-campo-de-kurt-lewin}}

Figura - Kurt Lewin (1890-1947) *.

\hypertarget{kurt-lewin-1890-1947}{%
\paragraph{Kurt Lewin (1890-1947)}\label{kurt-lewin-1890-1947}}

\begin{itemize}
\tightlist
\item
  Foi um psicólogo germano-estadunidense pioneiro da Psicologia
  Aplicada, Social e Organizacional nos Estados Unidos*.
\item
  Trabalhou 10 anos com os pioneiros da Gestalt: Max Wertheimer,
  Wolfgang Kohler e Kurt Koffka
\item
  Não era um Gestaltista, apesar dessa colaboração, já que ele seguiu um
  caminho teórico diferente desses pioneiros
\item
  Da colaboração com os pioneiros da Gestalt nasceu a \textbf{Teoria de
  Campo}
\item
  Lewin partiu da \textbf{teoria da Gestalt} para construir novos
  conhecimentos para a psicologia.

  \begin{itemize}
  \tightlist
  \item
    Ele abandonou a preocupação \textbf{psicofisiológica}
  \item
    Ele buscou na \textbf{Física} a base metodológica de sua psicologia.
  \end{itemize}
\end{itemize}

\hypertarget{o-conceito-de-espauxe7o-vital-e-de-campo-psicoluxf3gico-de-lewin}{%
\paragraph{\texorpdfstring{O Conceito de \textbf{Espaço Vital} e de
\textbf{Campo Psicológico} de
Lewin}{O Conceito de Espaço Vital e de Campo Psicológico de Lewin}}\label{o-conceito-de-espauxe7o-vital-e-de-campo-psicoluxf3gico-de-lewin}}

CAMPO VITAL

\begin{Shaded}
\begin{Highlighting}[]
\NormalTok{flowchart LR}

\NormalTok{A(Pessoa)\textless{}{-}{-}\textgreater{}B}
\NormalTok{C(Ambiente)\textless{}{-}{-}\textgreater{}B(((Relação com)))}
\end{Highlighting}
\end{Shaded}

\begin{itemize}
\tightlist
\item
  A \textbf{DEFINIÇÃO} de \textbf{ESPAÇO VITAL}: ``A \textbf{totalidade
  dos fatos} que DETERMINAM O COMPORTAMENTO do indivíduo, num certo
  momento''.
\item
  Outro conceito definido por Lewin foi o de \textbf{campo psicológico}:
  ``É o espaço vital considerado dinamicamente'' que deve ser
  considerado (1)tal como ele se apresenta para o indivíduo, (2)em um
  determinado momento.

  \begin{itemize}
  \tightlist
  \item
    Leva-se em conta:

    \begin{itemize}
    \tightlist
    \item
      O indivíduo
    \item
      O meio
    \item
      A \textbf{totalidade dos fatos} coexistentes e mutuamente
      interdependentes
    \end{itemize}
  \end{itemize}
\item
  O CAMPO PSICOLÓGICO \textbf{NÃO É} uma \textbf{realidade física}.
\item
  O CAMPO PSICOLÓGICO \textbf{É} uma \textbf{realidade fenomênica}.
\item
  Para Lewin, \textbf{NÃO SÃO APENAS} fatos físicos que produzem efetos
  sobre o \textbf{comportamento}
\item
  O CAMPO PSICOLÓGICO deve ser representado:

  \begin{itemize}
  \tightlist
  \item
    \textbf{(1) tal como ele existe para o indivíduo},
  \item
    \textbf{(2) num determinado momento}.
  \item
    Ele não existe de forma isolada e estática ( ``\ldots{} e não como
    ele é em si.'').
  \end{itemize}
\item
  São \textbf{ESSENCIAIS} para CONSTRUÇÃO DO CAMPO PSICOLÓGICO:

  \begin{itemize}
  \tightlist
  \item
    Os objetivos CONSCIENTES;
  \item
    Os objetivos INCONSCIENTES;
  \item
    Os sonhos;
  \item
    Os medos;
  \item
    As amizades;
  \item
    O AMBIENTE FÍSICO;
  \end{itemize}
\end{itemize}

\hypertarget{a-realidade-fenomuxeanica-em-kurt-lewin}{%
\subsubsection{A REALIDADE FENOMÊNICA EM KURT
LEWIN}\label{a-realidade-fenomuxeanica-em-kurt-lewin}}

O que é essa realidade fenomênica ?

\begin{enumerate}
\def\labelenumi{\arabic{enumi}.}
\tightlist
\item
  \textbf{A MANEIRA PARTICULAR COMO UM INDIVÍDUO INTERPRETA DETERMINADA
  SITUAÇÃO -\textgreater{} MEIO COMPORTAMENTAL da GESTALT}

  \begin{itemize}
  \tightlist
  \item
    \textbf{Obs}: ``A maneira particular como um indivíduo
    \textbf{INTERPRETA}'' significa a maneira como cada indivíduo
    \textbf{PERCEBE} enquanto fenômeno psicofisiológico
  \end{itemize}
\item
  As \textbf{CARACTERÍSTICAS DE PERSONALIDADE} do indivíduo;
\item
  Os \textbf{COMPONENTES EMOCIONAIS} ligados à situação \textbf{de vida}
  e \textbf{vivida} própria do indivíduo;
\item
  Os \textbf{COMPONENTES EMOCIONAIS} ligados \textbf{ao grupo} ao qual o
  indivíduo pertence;
\item
  As \textbf{SITUAÇÕES PASSADAS} que estejam ligadas ao acontecimento
\end{enumerate}

\begin{Shaded}
\begin{Highlighting}[]
\NormalTok{flowchart LR}
\NormalTok{A(\textless{}b\textgreater{}MANEIRA PARTICULAR\textless{}/b\textgreater{} COMO \textless{}br\textgreater{}\textless{}u\textgreater{}UM INDIVÍDUO\textless{}/u\textgreater{} \textless{}b\textgreater{}INTERPRETA\textless{}/b\textgreater{} \textless{}br\textgreater{}\textless{}u\textgreater{}DETERMINADA SITUAÇÃO\textless{}/u\textgreater{}){-}{-}\textgreater{}F}
\NormalTok{B(\textless{}b\textgreater{}CARACTERÍSTICAS DE\textless{}br\textgreater{}PERSONALIDADE\textless{}/b\textgreater{}\textless{}br\textgreater{}do indivíduo){-}{-}\textgreater{}F}
\NormalTok{C(\textless{}b\textgreater{}COMPONENTES EMOCIONAIS\textless{}/b\textgreater{}\textless{}br\textgreater{}ligados à \textless{}br\textgreater{}situação DE VIDA e VIVIDA\textless{}br\textgreater{} própria do indivíduo){-}{-}\textgreater{}F}
\NormalTok{D(\textless{}b\textgreater{}COMPONENTES EMOCIONAIS\textless{}/b\textgreater{}\textless{}br\textgreater{}ligados ao GRUPO\textless{}br\textgreater{}ao qual o indivíduo pertence){-}{-}\textgreater{}F}
\NormalTok{E(\textless{}b\textgreater{}SITUAÇÕES PASSADAS\textless{}/b\textgreater{}\textless{}br\textgreater{}que estejam ligadas\textless{}br\textgreater{}ao acontecimento){-}{-}\textgreater{}F(\textless{}b\textgreater{}ESPAÇO VITAL\textless{}/b\textgreater{}\textless{}br\textgreater{}\textquotesingle{}\textquotesingle{}Uma realidade fenomênica\textquotesingle{}\textquotesingle{},\textless{}br\textgreater{}segundo \textless{}b\textgreater{}KURT LEWIN\textless{}/b\textgreater{})}
\NormalTok{F{-}{-}\textgreater{}G(A \textless{}b\textgreater{}TOTALIDADE DOS FATOS\textless{}/b\textgreater{} que\textless{}br\textgreater{}determinam o comportamento\textless{}br\textgreater{}do indivíduo \textless{}b\textgreater{}num determinado momento\textless{}/b\textgreater{})}
\end{Highlighting}
\end{Shaded}

\hypertarget{exemplo-campo-psicoluxf3gico-e-espauxe7o-vital}{%
\subsubsection{EXEMPLO: Campo Psicológico e Espaço
Vital}\label{exemplo-campo-psicoluxf3gico-e-espauxe7o-vital}}

\begin{enumerate}
\def\labelenumi{\arabic{enumi}.}
\tightlist
\item
  \textbf{RELATO}:
\end{enumerate}

\begin{quote}
Um rapaz, ao chegar a sua casa, surpreende os pais num final de conversa
e escuta o seguinte: ``Ele chegou, é melhor não falarmos disso agora''.
\textbf{Ele entende que} OS PAIS CONVERSAVAM SOBRE UM \textbf{ASSUNTO
SÉRIO}, de que \textbf{ele não deveria tomar conhecimento}.
\textbf{RESOLVE não fazer nenhum comentário sobre o assunto}. Dias
depois, chegando novamente em casa, encontra seus pais na sala com dois
homens em ternos escuros. Imediatamente, associa esses homens ao final
da conversa escutada e entende que eles, de alguma forma, estariam
relacionados às preocupações dos pais.
\end{quote}

\begin{enumerate}
\def\labelenumi{\arabic{enumi}.}
\setcounter{enumi}{1}
\tightlist
\item
  \textbf{COMPORTAMENTO DETERMINADO PELO CAMPO PCISOLÓGICO}:
\end{enumerate}

\begin{itemize}
\tightlist
\item
  ``\textbf{RESOLVE} não fazer comentários sobre o assunto'';
\item
  Ele procurou ``\textbf{fingir que não havia escutado}'';
\end{itemize}

\begin{enumerate}
\def\labelenumi{\arabic{enumi}.}
\setcounter{enumi}{2}
\tightlist
\item
  \textbf{CONSIDERAÇÕES:}
\end{enumerate}

\begin{itemize}
\tightlist
\item
  Nessa estória, o \textbf{CAMPO PSICOLÓGICO} é representado pelas
  \textbf{``linhas de força''} que \textbf{(1/2)atraem a percepção} e
  \textbf{(2/2)lhe dão significado};
\item
  O rapaz (indivíduo) interpretou a situação pelo seu \textbf{ASPECTO
  FENOMÊNICO} e não pelo que ocorria de fato;
\item
  A INTERPRETAÇÃO ganhou CONSISTÊNCIA com a visita de duas pessoas que
  ele não conhecia (TOTALIDADE DOS FATOS). Isso foi possível porque o
  rapaz havia MEMORIZADO A SITUAÇÃO ANTERIOR e a ela ASSOCIADO A
  SITUAÇÃO SEGUINTE (a nova situações ganhou significado quando ligada a
  situação anterior);
\item
  O ESPAÇO VITAL é a SITUAÇÃO MAIS IMEDIATA ( A que DETERMINOU O
  COMPORTAMENTO );
\item
  O entendimento do \textbf{ESPAÇO VITAL} \textbf{depende diretamente}
  do \textbf{CAMPO PSICOLÓGICO}.
\end{itemize}

A compreensão do que Determina o Comportamento, segundo Kurt Lewin

\begin{Shaded}
\begin{Highlighting}[]
\NormalTok{flowchart LR}

\NormalTok{A(Compreensão do\textless{}br\textgreater{}\textless{}b\textgreater{}CAMPO PSICOLÓGICO\textless{}/b\textgreater{}){-}{-}\textgreater{}B(Compreensão do\textless{}br\textgreater{}\textless{}b\textgreater{}ESPAÇO VITAL\textless{}/b\textgreater{})}
\NormalTok{B{-}{-}\textgreater{}C(Compreensão do que\textless{}br\textgreater{}\textless{}b\textgreater{}DETERMINA O COMPORTAMENTO\textless{}/b\textgreater{})}
\end{Highlighting}
\end{Shaded}

\hypertarget{a-compreensuxe3o-do-conceito-de-grupo}{%
\subsubsection{A Compreensão do CONCEITO DE
GRUPO}\label{a-compreensuxe3o-do-conceito-de-grupo}}

\begin{itemize}
\tightlist
\item
  Praticamente todos os momentos de nossas vidas OCORREM dentro de
  GRUPOS;
\item
  Para Lewin:

  \begin{itemize}
  \tightlist
  \item
    A \textbf{CARACTERÍSTICA} ESSENCIALMENTE DEFINIDORA DE \textbf{GRUPO
    } é a \textbf{INTERDEPENDÊNCIA} de seus membros.
  \item
    Um grupo não é a \textbf{SOMA DE CARACTERÍSTICAS} de seus membros;
  \item
    Um grupo é \textbf{ALGO NOVO}, resultante dos \textbf{PROCESSOS QUE
    OCORREM} DENTRO DO GRUPO
  \item
    A MUNDANÇA DE UM MEMBRO PODE ALTERAR COMPLETAMENTE A DINÂMICA DO
    GRUPO;
  \end{itemize}
\item
  Os estudos de Lewin:

  \begin{itemize}
  \tightlist
  \item
    Deram ÊNFASE aos \textbf{PEQUENOS GRUPOS};
  \item
    Sobre os GRANDES GRUPOS: Ele considerava que a Psicologia não tinha
    INSTRUMENTAL SUCICIENTE para entender o estudo das GRANDES MASSAS;
  \end{itemize}
\end{itemize}

\hypertarget{o-conceito-de-campo-psicoluxf3gico-e-a-psicologia-social}{%
\subsubsection{O conceito de CAMPO PSICOLÓGICO e a PSICOLOGIA
SOCIAL}\label{o-conceito-de-campo-psicoluxf3gico-e-a-psicologia-social}}

\begin{itemize}
\tightlist
\item
  Lewin criou o conceito de CAMPO SOCIAL que é formado pelo GRUPO e pelo
  AMBIENTE;
\item
  Umas das CARACTERÍSTICAS DO GRUPO é o CLIMA SOCIAL;
\item
  Existe uma LIDERANÇA NO GRUPO

  \begin{itemize}
  \tightlist
  \item
    Tipos de LIDERANÇA:

    \begin{itemize}
    \tightlist
    \item
      Autocrática
    \item
      Democrática
    \item
      \emph{Leissez-faire}
    \end{itemize}
  \end{itemize}
\item
  Lewin pesquisou a DINÂMICA GRUPAL atrvés de um TRABALHO EXPERIMENTAL
  minucioso;
\item
  As contribuições de Kurt Lewin:

  \begin{itemize}
  \tightlist
  \item
    Estão presentes até hoje;
  \item
    Embasam:

    \begin{itemize}
    \tightlist
    \item
      OUTRAS TEORIAS que envolvem grupos;
    \item
      TÉCNICAS de trabalho com grupos
    \end{itemize}
  \end{itemize}
\end{itemize}

\hypertarget{capuxedtulo-5---a-psicanuxe1lise}{%
\subsection{Capítulo 5 - A
Psicanálise}\label{capuxedtulo-5---a-psicanuxe1lise}}

\begin{itemize}
\tightlist
\item
  Em breve, disponibilizaremos.
\end{itemize}

\hypertarget{capuxedtulo-7---a-psicologia-do-desenvolvimento}{%
\subsection{Capítulo 7 - A Psicologia do
Desenvolvimento}\label{capuxedtulo-7---a-psicologia-do-desenvolvimento}}

\begin{itemize}
\tightlist
\item
  Em breve, disponibilizaremos.
\end{itemize}

\hypertarget{livro-teorias-da-personalidade-feist-feist-roberts-2014}{%
\section{\texorpdfstring{Livro: \textbf{Teorias da Personalidade}
(FEIST; FEIST; ROBERTS,
2014)}{Livro: Teorias da Personalidade (FEIST; FEIST; ROBERTS, 2014)}}\label{livro-teorias-da-personalidade-feist-feist-roberts-2014}}

Figura -Livro \textbf{Teorias da Personalidade} (FEIST; FEIST; ROBERTS,
2014)

\hypertarget{capuxedtulo-9---maslow-teoria-holuxedstico-dinuxe2mica}{%
\subsection{Capítulo 9 - Maslow:
Teoria-Holístico-Dinâmica}\label{capuxedtulo-9---maslow-teoria-holuxedstico-dinuxe2mica}}

\begin{itemize}
\tightlist
\item
  Em breve, disponibilizaremos.
\end{itemize}

\hypertarget{cauxedtulo-10---rogers-teoria-centrada-na-pessoa}{%
\subsection{Caítulo 10 - Rogers: Teoria Centrada na
Pessoa}\label{cauxedtulo-10---rogers-teoria-centrada-na-pessoa}}

\begin{itemize}
\tightlist
\item
  Em breve, disponibilizaremos.
\end{itemize}

\hypertarget{livro-introduuxe7uxe3o-uxe0-psicologia-feldman-2015}{%
\section{\texorpdfstring{Livro: \textbf{Introdução à Psicologia}
(FELDMAN,
2015)\protect\hyperlink{feldman}{*}}{Livro: Introdução à Psicologia (FELDMAN, 2015)*}}\label{livro-introduuxe7uxe3o-uxe0-psicologia-feldman-2015}}

Figura Livro - FELDMAN, Robert S. \textbf{Introdução à Psicologia}.
10.ed. Porto Alegre: AMGH Editora, 2015

\hypertarget{capuxedtulo-1---muxf3dulo-3---livro-introduuxe7uxe3o-uxe0-psicologia}{%
\subsection{Capítulo 1 - Módulo 3 - Livro Introdução à
Psicologia}\label{capuxedtulo-1---muxf3dulo-3---livro-introduuxe7uxe3o-uxe0-psicologia}}

Capítulo 1 - Módulo 3 - Livro Introdução à Psicologia1

\begin{itemize}
\tightlist
\item
  Na pesquisa de arquivo, dados existentes são usados para testar uma
  hipótese, tais como:

  \begin{itemize}
  \tightlist
  \item
    Documentos censitários;
  \item
    Registros universitários;
  \item
    Recortes de jornal; etc.
  \end{itemize}
\item
  \textbf{Vantagens}:

  \begin{itemize}
  \tightlist
  \item
    É um meio econômico de testar hipóteses que alguém já coletou os
    dados básicos.
  \end{itemize}
\item
  \textbf{Desvantagens} do uso de dados já existentes:

  \begin{itemize}
  \tightlist
  \item
    Os dados podem não estar dispostos em uma forma que permita o
    pesquisador testar uma hipótese plenamente.
  \item
    As informações podem estar incompletas;
  \item
    As informações podem ter sido coletadas arbitrariamente;
  \item
    O que é mais comum: Os registros com as informações necessárias
    muitas vezes não existem;

    \begin{itemize}
    \tightlist
    \item
      Nesse caso, pode-se recorrer a outro método de pesquisa:
      \textbf{Observação naturalista}.
    \end{itemize}
  \end{itemize}
\end{itemize}

Capítulo 1 - Módulo 3 - Livro Introdução à Psicologia1

\begin{itemize}
\tightlist
\item
  O observador examina \textbf{um comportamento que ocorre naturalmente}
  e que \textbf{ele não interfere na situação};

  \begin{itemize}
  \tightlist
  \item
    O pesquisados simplesmente registra o que acontece
  \item
    O pesquisador não faz modificações na situação que está sendo
    observada;

    \begin{itemize}
    \tightlist
    \item
      Exemplo:

      \begin{itemize}
      \tightlist
      \item
        O pesquisador observa e registra o \textbf{tipo de ajuda
        prestada} em uma área urbana com alto índice de criminalidade.
      \end{itemize}
    \end{itemize}
  \item
    Vantagem: Obtemos uma amostra do que as pessoas fazem em seu
    \emph{habitat};
  \item
    ****INCONVENIENTES****:

    \begin{itemize}
    \tightlist
    \item
      A \textbf{impossibilidade de controlar} qualquer um dos
      \textbf{fatores de interesse};
    \item
      Poucos casos cuja previsibilidade permita observar dificultando a
      formulação de conclusões;
    \item
      É preciso esperar que as condições apropriadas (\textbf{fatores de
      interesse}) ocorram;
    \item
      Caso os participantes saibam ou percebam que estão sendo vigiados
      eles podem:

      \begin{itemize}
      \tightlist
      \item
        Alterar as suas reações;
      \item
        Produzir um comportamento que não é verdadeiramente
        representativo
      \end{itemize}
    \end{itemize}
  \end{itemize}
\end{itemize}

\begin{quote}
\begin{quote}
\begin{quote}
\begin{quote}
\begin{quote}
\begin{quote}
ACRESCENTAR LINK PARA SECAO DO RESUMO DOS LIVROS
\end{quote}
\end{quote}
\end{quote}
\end{quote}
\end{quote}
\end{quote}

Capítulo 1 - Módulo 3 - Livro Introdução à Psicologia1

\begin{itemize}
\tightlist
\item
  É um método simples e direto de conhecer, através de uma pergunta
  direta, o que as pessoas:

  \begin{itemize}
  \tightlist
  \item
    Pensam
  \item
    Sentem
  \item
    Fazem
  \end{itemize}
\item
  Através desse método, uma \textbf{amostra} de pessoas é escolhida para
  representar um \textbf{grupo de interesse} mais amplo. (\textbf{Uma
  população}), buscando-se conhecer:

  \begin{itemize}
  \tightlist
  \item
    Sobre seu \textbf{comportamento};
  \item
    Sobre seus \textbf{pensamentos};
  \item
    Sobre suas \textbf{Atitudes};
  \end{itemize}
\item
  Os pesquisadores conseguem ****deduzir**** com notável precisão como
  um grande grupo responderia;

  \begin{itemize}
  \tightlist
  \item
    Exemplos:

    \begin{itemize}
    \tightlist
    \item
      Pesquisadores que realizam \textbf{investigação sobre
      comportamento de ajuda}

      \begin{itemize}
      \tightlist
      \item
        Podem realizar uma pesquisa pedindo às pessoas que completem um
        \textbf{questionário} no qual elas indicam \textbf{sua
        relutância em prestar auxílio a alguém};
      \end{itemize}
    \item
      Pesquisadores interessados em \textbf{aprender sobre práticas
      sexuais}
    \item
      Podem realizar \textbf{levantamento} para \textbf{verificar quais
      práticas sexuais} \textbf{comuns} e quais \textbf{não são comuns}.
    \item
      Finalidade: Mapear as mudanças de noções de moralidade sexual
      durante as últimas décadas.
    \end{itemize}
  \end{itemize}
\item
  Desvantagens e armadilhas

  \begin{itemize}
  \tightlist
  \item
    É trabalhosa a \textbf{constituição de uma amostra estatísticamente
    representativa} (\textbf{amostra aleatória}) em que cada
    participante tenha a mesma chance (probabilidade) de ser incluído na
    amostra;
  \item
    Se a amostra não for \textbf{estatísticamente representativa} da
    população de interesse, os resultados da pesquisa terão pouco
    significado;
  \item
    Entrevistados podem não querer admitir:

    \begin{itemize}
    \tightlist
    \item
      Que tem comportamentos e/ou atitudes \textbf{socialmente
      indesejáveis};
    \item
      Que tem comportamento e/ou atitudes considerados por outras
      pessoas como \textbf{anormais};
    \item
      O que fazem em sua \textbf{intimidade};
    \end{itemize}
  \item
    Entrevistados podem \textbf{nem ter a consciência de quais são suas
    verdadeiras atitudes} ou porque elas as mantêm.
  \end{itemize}
\end{itemize}

Capítulo 1 - Módulo 3 - Livro Introdução à Psicologia1

\begin{itemize}
\tightlist
\item
  É uma investigação intensiva e em profundidade de \textbf{um único
  indivíduo} ou de \textbf{um pequeno grupo}.
\item
  Muitas vezes incluem \textbf{testagem psicológica}

  \begin{itemize}
  \tightlist
  \item
    É um procedimento em que um conjunto cuidadosamente elaborado de
    \textbf{instrumentos} é usado para \textbf{compreender algum
    }aspecto da personalidade** daquele indivíduo ou grupo**.
  \end{itemize}
\item
  Objetivos da realização de estudos de caso:

  \begin{itemize}
  \tightlist
  \item
    Aprender sobre os poucos indivíduos que estão sendo examinados;
  \item
    Usar conhecimentos adquiridos (a partir do estudo) para aperfeiçoar
    nossa compreensão das pessoas em geral.
  \end{itemize}
\item
  Comentários e curiosidades:

  \begin{itemize}
  \tightlist
  \item
    Sigmund Freud desenvolveu suas teorias por meio de \textbf{estudos
    de caso} de alguns de seus pacientes;
  \item
    Estudos de casos de terroristas podem \textbf{ajudar a identificar
    indivíduos que são propensos à violência};
  \end{itemize}
\item
  Desvantagens e inconvenientes:

  \begin{itemize}
  \tightlist
  \item
    Se os indivíduos examinados são excepcionais em algum aspecto, não é
    apropriado fazer generalizações para uma população mais ampla.
    \textgreater{} Observação: Mesmo em sua excepcionalidade, indivíduos
    ou pequenos grupos de indivíduos podem abrir caminho para
    \textbf{teorias} e \textbf{tratamentos} novos para
    \textbf{transtornos psicológicos}.
  \end{itemize}
\end{itemize}

Capítulo 1 - Módulo 3 - Livro Introdução à Psicologia1

De acordo com Feldman(2015, p.~31), os pesquisadores muitas vezes
desejam determinar a relação entre duas variáveis.

\begin{itemize}
\tightlist
\item
  ****Variáveis**** são \textbf{comportamentos}, \textbf{eventos} ou
  \textbf{outras características} que podem mudar ou variar de alguma
  maneira.

  \begin{itemize}
  \tightlist
  \item
    Exemplo: Uma pesquisa para verificar se a quantidade de estudo faz
    diferença nas notas em provas, as variáveis seriam \textbf{tempo de
    estudo} e \textbf{escores} em provas.
  \end{itemize}
\end{itemize}

Na pesquisa correlacional, dois conjuntos de variáveis são examinados
para determinar se eles estão associados ou ``correlacionados''.

\begin{itemize}
\tightlist
\item
  A \textbf{força} e a \textbf{direção} da relação entre as duas
  variáveis são representadas por uma estatística matemática conhecida
  como ****correlação**** (ou, mais formalmente, como ****coeficiente de
  correlação****), que pode variar de +1,0 a -1,0.
\item
  Uma \textbf{correlação positiva} indica que, à medida que uma variável
  aumenta, podemos prever que o valor da outra variável também
  aumentará.

  \begin{itemize}
  \tightlist
  \item
    Por exemplo, se previrmos que, quanto mais tempo os alunos passam
    estudando para uma prova, maiores serão suas notas e que, quanto
    menos eles estudam, menor serão suas pontuações nas provas, estamos
    esperando encontrar uma correlação positiva. (Valores mais altos da
    variável ``quantidade de tempo de estudo'' estariam associados a
    valores mais altos da variável ``pontuação na prova'', enquanto
    menores valores da ``quantidade de tempo de estudo'' estariam
    associados a valores mais baixos da variável ``pontuação na
    prova''.)
  \item
    A correlação, então, seria indicada por um número positivo e, quanto
    mais forte for a associação entre estudar e pontuação nas provas,
    mais próximo de 1,0 será o número.
  \end{itemize}
\item
  Em contraste, uma \textbf{correlação negativa} demonstra que, à medida
  que o valor de uma variável aumenta, o valor de outra diminui. Por
  exemplo, podemos prever que, à medida que o número de horas passadas
  estudando aumenta, o número de horas passadas participando de festas
  diminui. Nesse caso, estamos esperando uma correlação negativa, que
  varia entre 0 e -1,0.

  \begin{itemize}
  \tightlist
  \item
    Mais estudo está associado a menor participação em festas, enquanto
    menos estudo está associado a maior participação em festas. Quanto
    mais forte for a associação entre estudar e participar de festas,
    mais próxima será a correlação de -1,0. Por exemplo, uma correlação
    de -0,85 indicaria uma associação negativa forte entre participar de
    festas e estudar.
  \end{itemize}
\item
  Evidentemente, é bem possível que exista \textbf{pouca ou nenhuma
  relação} entre duas variáveis.

  \begin{itemize}
  \tightlist
  \item
    Por exemplo, provavelmente \textbf{não esperaríamos encontrar uma
    relação} entre o \textbf{número de horas de estudo} e
    \textbf{altura}. A ausência de uma relação seria indicada por uma
    correlação próxima de 0. Por exemplo, se encontrássemos uma
    correlação de -0,02 ou 0,03, isso indicaria que existe praticamente
    nenhuma associação entre duas variáveis: saber o quanto alguém
    estuda nos diz nada sobre sua altura.
  \end{itemize}
\item
  Quando duas variáveis estão fortemente correlacionadas, somos tentados
  a presumir que uma variável causa a outra. Por exemplo, se
  descobrirmos que mais estudo está associado a notas mais altas,
  podemos supor que estudar mais causa notas mais altas. Embora esta não
  seja uma má suposição, ela continua sendo apenas uma suposição --
  porque descobrir que suas variáveis estão correlacionadas não
  significa que exista uma relação causal entre elas.

  \begin{itemize}
  \tightlist
  \item
    A \textbf{correlação forte} sugere que saber o quanto uma pessoa
    estuda pode ajudar-nos a prever como aquela pessoa vai se sair em
    uma prova, mas isso não significa que estudar causa o desempenho na
    prova.

    \begin{itemize}
    \tightlist
    \item
      Em vez disso, por exemplo, as \textbf{pessoas que estão mais
      interessadas no assunto} poderiam estudar mais do que aquelas que
      estão menos interessadas; assim, a \textbf{quantidade de
      interesse}, e não o número de horas passadas estudando, preveria o
      desempenho em provas.
    \end{itemize}
  \item
    O simples fato de que \textbf{duas variáveis ocorrem juntas}
    \textbf{não significa que uma cause a outra}. Simplesmente fornecem
    uma medida da força da relação entre duas variáveis

    \begin{itemize}
    \tightlist
    \item
      Exemplos:

      \begin{itemize}
      \tightlist
      \item
        Suponha que você descobriu que o \textbf{número de lugares de
        prática religiosa} em uma grande amostra de cidades estava
        positivamente relacionado ao \textbf{número de pessoas detidas},
        significando que, quanto mais lugares de prática religiosa, mais
        detenções havia em uma cidade. Isso significa que a presença de
        mais espaços de prática religiosa causou o maior número de
        detenções? Quase certamente não, é claro. \textbf{Nesse caso, a
        causa subjacente é o tamanho da cidade}: em cidades maiores,
        existem mais espaços de prática religiosa tanto como de mais
        detenções.
      \item
        Crianças que assistem a muitos programas de televisão com alto
        nível de agressão são propensas a demonstrar um grau
        relativamente alto de comportamento agressivo e que aquelas que
        assistem menos a programas de televisão que retratam agressão
        são inclinadas a exibir um grau relativamente baixo desse
        comportamento (ver Fig. 2). Contudo, \textbf{não podemos dizer
        que a agressão é causada por ver televisão}, pois \textbf{muitas
        outras explicações são possíveis}. Pessoas que já são altamente
        agressivas poderiam escolher ver programas com um alto conteúdo
        agressivo porque elas são agressivas
      \end{itemize}
    \end{itemize}
  \end{itemize}
\item
  Desvantagem da pesquisa correlacional:

  \begin{itemize}
  \tightlist
  \item
    A \textbf{impossibilidade} de a pesquisa correlacional
    \textbf{demonstrar relações de causa e efeito} é uma desvantagem
    crucial para seu uso. Contudo, ****existe uma técnica alternativa
    que estabelece causalidade: o experimento\textbf{.}
  \end{itemize}
\item
  ****EXPERIMENTO****: Investigação da relação entre duas (ou mais)
  variáveis alterando--se deliberadamente uma situação e observando-se
  os efeitos dessa alteração em outros aspectos da situação.
\end{itemize}

\hypertarget{muxe9todo-cientuxedfico-muxe9todo-experimental}{%
\subsubsection{Método Científico: Método
Experimental}\label{muxe9todo-cientuxedfico-muxe9todo-experimental}}

\hypertarget{pesquisa-experimental}{%
\paragraph{Pesquisa Experimental}\label{pesquisa-experimental}}

Slides

\begin{itemize}
\tightlist
\item
  Os estudo experimental é o método para estabelecer explicações, de
  demonstrar a relação de causa e efeito.
\item
  \textbf{HIPÓTESE}: É uma afirmação. Os experimentos iniciam se com uma
  hipótese , acerca dos eventos, conhecidos como variáveis.
\item
  A essência de um Experimento:

  \begin{itemize}
  \tightlist
  \item
    Os pesquisadores deliberadamente manipulam a variável independente o
    evento cuja influência está sendo investigada
  \item
    Eles pedem que variáveis estranhas ou irrelevantes afetem os
    resultados do estudo
  \item
    Eles medem os efeitos da manipulação sobre a variável dependente
  \end{itemize}
\item
  Exemplo:

  \begin{itemize}
  \tightlist
  \item
    \textbf{Objetivo}: Será que o perfume desencadeia vivências
    nostálgicas e aumenta afetos positivos, autoestima e conexão social?
  \item
    \textbf{Hipótese}: Os participantes na condição experimental
    apresentará níveis mais elevados de afetos positivos, autoestima e
    apoio social
  \item
    \textbf{Método}: Participaram 160 pessoas. Metade para a condição
    experimental (12 perfumes) e outra metade condição controle (sem
    exposição do perfume)
  \item
    \textbf{Resultados}: Os participantes na condição experimental,
    apresentou mais afetos positivos, autoestima e apoio social,
    comparado ao grupo controle
  \end{itemize}
\end{itemize}

Capítulo 1 - Módulo 3 - Livro Introdução à Psicologia1

\begin{itemize}
\tightlist
\item
  Em um experimento formal, o pesquisador investiga a relação entre duas
  (ou mais) variáveis \textbf{alterando deliberadamente uma variável} em
  uma \textbf{situação controlada} e \textbf{observando os efeitos
  daquela mudança} em \textbf{outros aspectos da situação}.
\item
  Em um experimento, portanto, as ****condições**** são \textbf{criadas}
  e \textbf{controladas} pelo pesquisador, que deliberadamente faz uma
  alteração nessas condições a fim de observar os efeitos daquela
  mudança.
\item
  A alteração que o pesquisador deliberadamente faz em um experimento é
  denominada ****manipulação experimental****. Manipulações
  experimentais são ****usadas para detectar relações entre diferentes
  variáveis**** (Staub, 2011 apud Feldman, 2015).
\item
  A ****realização de um experimento**** envolve \textbf{várias etapas},
  mas o processo geralmente se \textbf{inicia} com o
  \textbf{desenvolvimento de uma ou mais hipóteses} a serem testadas
  pelo experimento.

  \begin{itemize}
  \tightlist
  \item
    Por exemplo, \textbf{Latané} e \textbf{Darley}, ao testar sua
    \textbf{teoria} acerca da \textbf{difusão da responsabilidade no
    comportamento de espectadores}, elaboraram a seguinte hipótese:
    quanto maior for o número de pessoas que testemunham uma situação de
    emergência, menor será a probabilidade de que alguma delas ajude a
    vítima.
  \item
    Eles então ****criaram um experimento**** para testar essa hipótese.

    \begin{itemize}
    \tightlist
    \item
      O \textbf{primeiro passo} foi formular uma ****definição
      operacional da hipótese****, conceitualizando-a de um modo que ela
      pudesse ser testada.
    \item
      Latané e Darley tiveram de levar em conta o ****princípio
      fundamental da pesquisa experimental**** mencionado anteriormente:
      os experimentadores \textbf{devem manipular ao menos uma variável
      a fim de observar os efeitos da manipulação em outra variável},
      enquanto outros fatores na situação são mantidos constantes.
    \item
      Entretanto, ****a manipulação não pode ser vista isoladamente****;
      para que uma relação de causa e efeito seja estabelecida, os
      efeitos da manipulação devem ser comparados com os efeitos de
      nenhuma manipulação ou de outro tipo de manipulação.
    \end{itemize}
  \end{itemize}
\end{itemize}

\hypertarget{grupos-experimentais-e-grupos-controle}{%
\subparagraph{Grupos experimentais e
grupos-controle}\label{grupos-experimentais-e-grupos-controle}}

\begin{itemize}
\tightlist
\item
  A pesquisa experimental exige, portanto, que as respostas de ao menos
  dois grupos sejam comparadas.
\item
  ****Um grupo**** receberá um tratamento especial -- a manipulação
  implementada pelo experimentador -- e ****outro grupo**** receberá um
  tratamento diferente ou nenhum tratamento.
\item
  Qualquer grupo que recebe um tratamento é denominado ****grupo
  experimental****
\item
  Um grupo que não recebe tratamento é denominado ****grupo-controle****
\item
  Em alguns experimentos, existem vários grupos experimentais e de
  controle, cada um dos quais comparado com outro grupo.
\item
  \textbf{Empregando} \textbf{grupos experimentais} e de
  \textbf{controle} em um experimento, os pesquisadores ****são capazes
  de descartar**** a \textbf{possibilidade de que alguma outra variável
  que não a manipulação experimental tenha produzido os resultados
  observados no experimento}.
\item
  Sem um grupo-controle, ****não poderíamos ter certeza**** de que
  alguma outra variável, como, por exemplo, a temperatura no momento da
  execução do experimento, a cor de cabelo do experimentador ou mesmo a
  mera passagem do tempo, não estava causando as mudanças observadas.

  \begin{itemize}
  \tightlist
  \item
    Por exemplo:

    \begin{itemize}
    \tightlist
    \item
      Considere um pesquisador de medicina que acredita que inventou um
      medicamento que cura o resfriado.
    \item
      Para testar sua alegação, ele administra o remédio um dia a um
      grupo de 20 pessoas que estão resfriadas e descobre que 10 dias
      depois todas elas estão curadas.
    \item
      Eureca? Mais devagar. Um observador que considere esse estudo
      falho poderia argumentar sensatamente que as pessoas teriam
      melhorado mesmo sem o medicamento.
    \item
      O que o pesquisador evidentemente precisava era de um
      grupo-controle formado por pessoas resfriadas que não recebem o
      remédio e cuja saúde também é verificada 10 dias depois.
    \end{itemize}
  \end{itemize}
\item
  Somente quando existe uma diferença significativa entre grupos
  experimental e de controle é que a eficácia do remédio pode ser
  avaliada. ****Ao usar grupos-controle****, então, os pesquisadores
  podem isolar causas específicas para seus achados -- e extrair
  inferências de causa e efeito.
\item
  Voltando ao experimento de Latané e Darley:

  \begin{itemize}
  \tightlist
  \item
    Vemos que os pesquisadores recisavam traduzir sua hipótese para algo
    que pudesse ser testado.
  \item
    Para fazer isso, decidiram criar uma falsa situação de emergência
    que pareceria requerer a ajuda de um espectador.
  \item
    Como manipulação experimental, optaram por \textbf{variar o número
    de espectadores presentes}.
  \item
    Eles poderiam ter usado apenas \textbf{um grupo experimental}, por
    exemplo, de duas pessoas presentes, e \textbf{um grupo-controle} com
    apenas uma pessoa presente para fins de comparação.
  \item
    Em vez disso, optaram por um procedimento mais complexo envolvendo a
    \textbf{criação de grupos de três tamanhos} -- compostos por
    \textbf{duas}, \textbf{três} e \textbf{seis} pessoas -- ****que
    poderiam ser comparados um com o outro****.
  \end{itemize}
\end{itemize}

\hypertarget{variuxe1veis-independentes-e-dependentes}{%
\subparagraph{Variáveis Independentes e
Dependentes}\label{variuxe1veis-independentes-e-dependentes}}

\begin{itemize}
\tightlist
\item
  O projeto experimental de Latané e Darley agora incluía uma definição
  experimental do que é chamado de variável independente.
\item
  ****A variável independente**** é a condição que é manipulada por um
  experimentador.

  \begin{itemize}
  \tightlist
  \item
    Você pode pensar a variável independente como sendo independente das
    ações daqueles que participam de um experimento; ela é controlada
    pelo experimentador.
  \item
    No caso do experimento de Latané e Darley, a variável independente
    era o número de pessoas presentes, que foi manipulado pelos
    experimentadores.
  \end{itemize}
\item
  O próximo passo era decidir como eles determinariam o efeito que o
  número variável de espectadores tinha no comportamento das pessoas no
  experimento.
\item
  ****Fundamental em todo experimento é a VARIÁVEL DEPENDENTE****,
  aquela que é medida e que se espera que mude em função das alterações
  provocadas pelo experimentador manipulando a \textbf{variável
  independente}.
\item
  ****A variável dependente**** é dependente das ações dos participantes
  ou sujeitos -- as pessoas que participam de um experimento.

  \begin{itemize}
  \tightlist
  \item
    Latané e Darley tinham várias escolhas possíveis para sua medida
    dependente.
  \item
    Uma poderia ter sido uma simples verificação (sim/não) do
    comportamento de ajuda dos participantes.
  \item
    Contudo, os investigadores também queriam uma análise mais precisa
    do comportamento de ajuda.
  \item
    Consequentemente, também mediram a quantidade de tempo que levava
    para um participante prestar ajuda. Latané e Darley agora tinham
    todos os componentes necessários de um experimento.
  \end{itemize}
\item
  ****A variável independente****, manipulada por eles, era o
  \textbf{número de espectadores presentes} em uma situação de
  emergência.
\item
  ****A variável dependente**** era verificar se \textbf{os espectadores
  em cada um dos grupos prestavam ajuda} e a \textbf{quantidade de
  tempo} que eles levavam para isso.
\item
  Consequentemente, como todos os experimentos, esse teve tanto uma
  variável independente como uma variável dependente.
\item
  \textbf{Todos os verdadeiros experimentos em psicologia}
  \textbf{encaixam-se nesse modelo simples}.
\end{itemize}

\begin{quote}
TERMOS IMPORTANTES: \textgreater{} \textbf{Variável Independente}:
Variável que \textbf{é manipulada} por um experimentador. \textgreater{}
\textbf{Variável Dependente}: Variável que \textbf{é mensurada }e que
\textbf{se espera que se modifique} como resultado de mudanças causadas
pela manipulação da variável independente realizada pelo experimentador.
\end{quote}

\hypertarget{distribuiuxe7uxe3o-aleatuxf3ria-dos-participantes}{%
\subparagraph{Distribuição aleatória dos
participantes}\label{distribuiuxe7uxe3o-aleatuxf3ria-dos-participantes}}

\begin{itemize}
\tightlist
\item
  Para tornar o experimento um teste válido da hipótese, Latané e Darley
  precisavam adicionar um passo final ao projeto experimental: designar
  corretamente os participantes a determinado grupo experimental.
\item
  O significado desse passo torna-se claro quando examinamos vários
  procedimentos alternativos. Por exemplo, os experimentadores poderiam
  ter designado somente homens para o grupo com dois espectadores,
  apenas mulheres para o grupo com três espectadores e ambos (homens e
  mulheres) para o grupo com seis espectadores. Contudo, caso tivessem
  feito isso, as eventuais diferenças que encontraram no comportamento
  de ajuda não poderiam ser atribuídas com certeza unicamente ao tamanho
  do grupo, pois elas poderiam igualmente ter resultado da composição do
  grupo. Um procedimento mais sensato seria assegurar que cada grupo
  tivesse a mesma composição em termos de gênero; assim, os
  pesquisadores teriam podido fazer comparações entre os grupos com mais
  precisão.
\item
  Os participantes em cada um dos grupos experimentais devem ser
  comparáveis, e é muito fácil criar grupos que sejam semelhantes em
  termos de gênero. No entanto, o problema torna-se um pouco mais
  traiçoeiro quando consideramos outras características dos
  participantes. Como podemos garantir que os participantes em cada
  grupo experimental serão igualmente inteligentes, extrovertidos,
  cooperativos, e assim por diante, quando a lista de características --
  qualquer uma poderia ser importante -- é potencialmente infinita?
\item
  A solução é um procedimento simples, mas elegante, chamado de
  designação aleatória à condição.
\item
  Os participantes são designados para diferentes grupos experimentais,
  ou ``condições'', com base no acaso e somente no acaso. O
  experimentador poderia, por exemplo, fazer um sorteio com moeda para
  cada participante e designar um participante para um grupo quando
  desse ``cara'' e para outro grupo quando desse ``coroa''. A vantagem
  dessa técnica é que existe uma chance idêntica de que as
  características dos participantes se distribuirão entre os diversos
  grupos. Quando um pesquisador emprega distribuição aleatória -- o que
  na prática geralmente é realizado usando números aleatórios gerados
  por computador --, é provável que cada um dos grupos terá
  aproximadamente a mesma proporção de pessoas inteligentes,
  cooperativas, extrovertidas, do sexo masculino e feminino, e assim por
  diante.
\end{itemize}

Figura - Efeitos da substância \textbf{propanolol} (figura 3)

\begin{itemize}
\tightlist
\item
  A \textbf{Figura 3} do livro de FELDMAN apresenta outro exemplo de um
  experimento. Como todos os experimentos, esse inclui um conjunto de
  elementos-chave, que você deve lembrar ao considerar se um estudo
  científico é realmente um experimento.

  \begin{itemize}
  \tightlist
  \item
    Uma variável independente, a variável que é manipulada pelo
    experimentador.
  \item
    Uma variável dependente, a variável que é medida pelo experimentador
    e que se espera que mude como resultado da manipulação da variável
    independente.
  \item
    Um procedimento que distribui aleatoriamente os participantes em
    diferentes grupos experimentais, ou ``condições'', da variável
    dependente.
  \item
    Uma hipótese que prevê o efeito que a variável independente terá na
    variável dependente. Somente se todos esses elementos estiverem
    presentes é que um estudo científico pode ser considerado um
    experimento verdadeiro em que relações de causa e efeito podem ser
    determinadas. (Para um resumo dos diferentes tipos de pesquisa que
    discutimos, ver \textbf{Fig. 4} do livro de FELDMAN .)
  \end{itemize}
\end{itemize}

Figura - Estratégias de pesquisa (figura 4)

\hypertarget{outros-tuxf3picos-nuxe3o-abordados}{%
\subparagraph{Outros tópicos não
abordados}\label{outros-tuxf3picos-nuxe3o-abordados}}

No capítulo 3 do livro de FELDMAN, ainda constam dois tópicos que não
abordei aqui por terem baixa chance de serem explorados na prova o que
não prejudica a compreenção da \textbf{Pesquisa Experimental}

\begin{itemize}
\tightlist
\item
  Latané e Darley estavam certos?
\item
  Indo além do estudo
\end{itemize}

\hypertarget{livro-introduuxe7uxe3o-uxe0-psicologia-davidoff-2001}{%
\section{\texorpdfstring{Livro: \textbf{Introdução à Psicologia}
(DAVIDOFF,
2001)}{Livro: Introdução à Psicologia (DAVIDOFF, 2001)}}\label{livro-introduuxe7uxe3o-uxe0-psicologia-davidoff-2001}}

Figura - Livro Introdução à Psicologia (DAVIDOFF, 2001)

\hypertarget{capuxedtulo-xx}{%
\subsection{Capítulo XX}\label{capuxedtulo-xx}}

\begin{itemize}
\tightlist
\item
  A definir.
\end{itemize}

\hypertarget{referuxeancias-bibliogruxe1ficas-1}{%
\section{Referências
Bibliográficas}\label{referuxeancias-bibliogruxe1ficas-1}}

BOCH, Ana Mercês Bahia; FURTADO, Odair; TEIXEIRA, Maria de Lourdes
Trassi. Psicologias: Uma Introdução ao Estudo da Psicologia. 13.ed. São
Paulo: Saraiva, 2001.

DAVIDOFF, Linda L. \textbf{Introdução à Psicologia}. 3.ed. São
Paulo:Pearson, 2001

FEIST, Jess; FEIST, Gregory J.; ROBERTS, Tomi-Ann. \textbf{Teorias da
Personalidade}. 8.ed. Porto Alegre:AMGH, 2014

FELDMAN, Robert S. \textbf{Introdução à Psicologia}. 10.ed. Porto
Alegre: AMGH Editora, 2015

KURT LEWIN. In: WIKIPÉDIA: a enciclopédia livre. Wikimedia, 2022.
Disponível em:
\textless{}\url{https://en.wikipedia.org/wiki/Kurt_Lewin}\textgreater.
Acesso em: 30 set. 2022.

\hypertarget{p1---histuxf3ria-da-psicologia}{%
\chapter{P1 - História da
Psicologia}\label{p1---histuxf3ria-da-psicologia}}

Neste capítulo estarão contidos os resumos relacionados com a disciplina

Em breve\ldots{}

\hypertarget{p1---introduuxe7uxe3o-uxe0-filosofia}{%
\chapter{P1 - Introdução à
Filosofia}\label{p1---introduuxe7uxe3o-uxe0-filosofia}}

Neste capítulo estarão contidos os resumos dos slides da disciplina
Introdução à Filosofia.

\hypertarget{livro-introduuxe7uxe3o-uxe0-filosofia---chaui---2.ed.---2013}{%
\section{Livro: Introdução à Filosofia - CHAUI - 2.ed. -
2013}\label{livro-introduuxe7uxe3o-uxe0-filosofia---chaui---2.ed.---2013}}

\begin{figure}

{\centering \includegraphics[width=0.5\linewidth]{imagens/Capa-Livro-Introducao-a-Filosofia-MARILENA-CHAUI} 

}

\caption{Livro <b>Introdução à filosofia</b> - CHAUI - 2.ed. - 2013}\label{fig:unnamed-chunk-5}
\end{figure}

\begin{itemize}
\tightlist
\item
  Resumo elaborado para o trabalho de História da Psicologia: ``Mude
  minha Opinião''.
\item
  Tema do debate: ``O comportamento humano pode ser controlado'';
\end{itemize}

\hypertarget{capuxedtulo-28---a-liberdade}{%
\subsection{Capítulo 28 - A
Liberdade}\label{capuxedtulo-28---a-liberdade}}

Esse resumo foi elaborado para apresentação de atividade da disciplina
História da Psicologia para segunda nota avaliativa.

\hypertarget{introduuxe7uxe3o}{%
\subsubsection{Introdução}\label{introduuxe7uxe3o}}

\begin{itemize}
\tightlist
\item
  Como é possível SER LIVRE ?

  \begin{itemize}
  \tightlist
  \item
    Se nossa vida transcorre em meio à de outrops indivíduos;
  \item
    Se nossa vida transcorre em meio a instituições sociais;
  \item
    Se nossa vida transcorre em meio a normas culturais;
  \item
    Se nossa vida transcorre em meio às forças da natureza.
  \end{itemize}
\item
  A LIBERDADE é OBJETO CENTRAL dos estudos da ÉTICA como disciplina
  filosófica;
\item
  A ética dedica-se:

  \begin{itemize}
  \tightlist
  \item
    A DEFINIR e ANALISAR os \textbf{elementos} que (1)
    \textbf{possibilitam} ou (2) \textbf{impedem} a LIBERDADE;
  \item
    A tratar de duas grandes questões:

    \begin{itemize}
    \tightlist
    \item
      Há limites para a LIBERDADE ?
    \item
      Como e em que termos ela pode ser INTEGRALMENTE CONQUISTADA por
      todos ?
    \end{itemize}
  \end{itemize}
\end{itemize}

\hypertarget{a-liberdade-como-problema}{%
\subsubsection{A Liberdade como
PROBLEMA}\label{a-liberdade-como-problema}}

\begin{itemize}
\tightlist
\item
  Chaiu (2013, p.~278) apresenta duas questões:

  \begin{itemize}
  \tightlist
  \item
    Qual o núcleo da LIBERDADE ?
  \item
    Como podemos sentir a AUSÊNCIA DE LIBERDADE ?
  \end{itemize}
\item
  Ela inicia o diálogo para abordar esses dois problemas apresentando:

  \begin{itemize}
  \tightlist
  \item
    Um POEMA de José Paulo Paes (``O melhor poeta da minha rua'')
  \item
    Um POEMA de Carlos Drummond de Andrade (``Sete faces'')
  \end{itemize}
\item
  Os dois poemas apontam para o grande tema da ética:

  \begin{itemize}
  \tightlist
  \item
    O que está ?
  \item
    O que não está em nosso poder ?
  \item
    Até onde se estende o PODER:

    \begin{itemize}
    \tightlist
    \item
      Da nossa VONTADE ?
    \item
      Do nosso DESEJO ?
    \item
      Da nossa CONSCIÊNCIA ?
    \end{itemize}
  \end{itemize}
\item
  Por fim, até onde se estende o PODER DA NOSSA LIBERDADE ?

  \begin{itemize}
  \tightlist
  \item
    O que ESTÁ EM NOSSO PODER ?
  \item
    O que DEPENDE INTEIRAMENTE DE \textbf{CAUSAS} e \textbf{FORÇAS}
    EXTERIORES ?
  \end{itemize}
\item
  Mais um poema é apresentado pela autora para consolidar seu
  raciocínio: Vicente de Carvalho (``Velho tema'')
\item
  Os três poetas nos colocam diante da LIBERDADE como PROBLEMA, seja:

  \begin{itemize}
  \tightlist
  \item
    De modo pessimista (como em José Paulo Paes e Vicente de Carvalho)
  \item
    De modo otimista (como em Carlos Drummond)
  \end{itemize}
\end{itemize}

\hypertarget{a-liberdade-como-questuxe3o-filosuxf3fica}{%
\subsubsection{A Liberdade como QUESTÃO
FILOSÓFICA}\label{a-liberdade-como-questuxe3o-filosuxf3fica}}

\begin{itemize}
\tightlist
\item
  A questão da liberdade se apresenta na forma de DOIS PARES OPOSTOS:

  \begin{itemize}
  \tightlist
  \item
    O par necessidade-liberdade

    \begin{itemize}
    \tightlist
    \item
      NECESSIDADE é o termo empregado para se referir ao todo da
      realidade, existente em si e por si, que age sem nós e nos insere
      em sua rede de causas e efeitos, condições e consequências.
    \end{itemize}
  \item
    O par contingência-liberdade

    \begin{itemize}
    \tightlist
    \item
      CONTINGÊNCIA ou ACASO significam que a realidade é imprevisível e
      mutável, impossibilitando deliberação e decisão racionais,
      definidoras da liberdade.
    \item
      Num mundo onde tudo acontece por acidente, somos como um frágil
      barquinho perdido num mar tempestuoso, levado em todas as
      direções, ao sabor das vagas e dos ventos.
    \end{itemize}
  \end{itemize}
\item
  FATALIDADE é o termo usado quando pensamos em forças transcendentes
  superiores às nossas e que nos governam, quer o queiramos, quer não.
\item
  DETERMINISMO é o termo empregado, a partir do século XIX, para se
  referir às relações causais necessárias que regem a realidade
  conhecida e controlada pela ciência;

  \begin{itemize}
  \tightlist
  \item
    No caso da ética, refere-se ao ser humano como objeto das ciências
    naturais (química e biologia) e das ciências humanas (sociologia e
    psicologia).
  \item
    Portanto, subordina-o completamente a leis e causas que condicionam
    seus pensamentos, sentimentos e ações, tornando a liberdade
    ilusória.
  \end{itemize}
\item
  O que poderia estar em nosso poder?
\item
  Necessidade, fatalidade, determinismo

  \begin{itemize}
  \tightlist
  \item
    Significam que não há lugar para a liberdade, porque o curso das
    coisas e de nossa vida já está fixado, sem que nele possamos
    intervir;
  \end{itemize}
\item
  Contingência e acaso

  \begin{itemize}
  \tightlist
  \item
    Significam que não há lugar para a liberdade, porque não há curso
    algum das coisas e de nossa vida sobre o qual pudéssemos intervir.
  \end{itemize}
\end{itemize}

\hypertarget{truxeas-grandes-concepuxe7uxf5es-filosuxf3ficas-da-liberdade}{%
\subsubsection{Três grandes CONCEPÇÕES FILOSÓFICAS DA
LIBERDADE}\label{truxeas-grandes-concepuxe7uxf5es-filosuxf3ficas-da-liberdade}}

\begin{itemize}
\tightlist
\item
  Na mitologia grega NECESSIDADE e CONTINGÊNCIA

  \begin{itemize}
  \tightlist
  \item
    Necessidade: As Moiras ( conhecidas também como Parcas ) eram as
    três irmãs que determinavam o destino (FATALIDADE), tanto dos
    deuses, quanto dos seres humanos.
  \item
    A contingência (ou o acaso): Era representada pela Fortuna, mulher
    volúvel e caprichosa, que trazia nas mãos uma roda, fazendo-a girar
    de tal modo que quem estivesse no alto (a boa fortuna ou boa sorte)
    caísse (infortúnio ou má sorte) e quem estivesse embaixo fosse
    elevado.

    \begin{itemize}
    \tightlist
    \item
      INCONSTANTE, INCERTA e CEGA: a RODA DA FORTUNA era a pura sorte,
      boa ou má, contra a qual nada se poderia fazer;
    \end{itemize}
  \end{itemize}
\item
  As \textbf{TEORIAS ÉTICAS} procuraram sempre enfrentar o duplo
  problema da necessidade e da contingência, definindo o campo da
  liberdade possível
\end{itemize}

\hypertarget{primeira-as-concepuxe7uxf5es-de-aristuxf3teles-e-de-satre}{%
\paragraph{PRIMEIRA: As concepções de Aristóteles e de
Satre}\label{primeira-as-concepuxe7uxf5es-de-aristuxf3teles-e-de-satre}}

\begin{itemize}
\tightlist
\item
  ARISTÓTELES

  \begin{itemize}
  \tightlist
  \item
    Postulou a \textbf{PRIMEIRA} grande \textbf{TEORIA FILOSÓFICA DA
    LIBERDADE} (Livro Ética a Nicômaco)
  \item
    Nesse sentido, a LIBERDADE se opõe ( Necessidade x Contingência )

    \begin{itemize}
    \tightlist
    \item
      Ao que é CONDICIONÁDO EXTERNAMENTE (Necessidade)
    \item
      Ao que ACONTECE SEM ESCOLHA DELIBERADA (Contingência)
    \end{itemize}
  \item
    Afirma que ``É livre aquele que tem em si mesmo o PRINCÍPIO para
    AGIR ou NÃO AGIR'';
  \item
    LIBERDADE:

    \begin{itemize}
    \tightlist
    \item
      É o poder PLENO e INCONDICIONAL da VONTADE para determinar a si
      mesmo (\textbf{AUTODETERMINAÇÃO});
    \item
      É uma CAPACIDADE que

      \begin{enumerate}
      \def\labelenumi{\alph{enumi}.}
      \tightlist
      \item
        Não encontra \textbf{obstáculos} para se realizar;
      \item
        Nem é \textbf{forçada} por \textbf{coisa alguma} para agir;
      \end{enumerate}
    \end{itemize}
  \item
    Contingência ( Puro acaso ) x Possível ( Pode acontecer desde que o
    ser humano DELIBERE e DECIDA realizar uma ação )
  \item
    LIBERDADE:

    \begin{itemize}
    \tightlist
    \item
      É o princípio para escolher entre \textbf{ALTERNATIVAS POSSÍVEIS}
    \item
      Realiza-se:

      \begin{enumerate}
      \def\labelenumi{\alph{enumi}.}
      \tightlist
      \item
        Decisão
      \item
        Ato Voluntário
      \end{enumerate}
    \end{itemize}
  \end{itemize}
\item
  Contrariamente à necessidade e à contingência, sob as quais o agente
  sofre a ação de uma CAUSA EXTERNA que o OBRIGA A AGIR de determinada
  maneira no ATO VOLUNTÁRIO LIVRE o agente é causa de si, isto é, CAUSA
  INTEGRAL DE SUA AÇÃO.
\item
  Sem dúvida, seria possível dizer que a VONTADE LIVRE é determinada:

  \begin{itemize}
  \tightlist
  \item
    Pela RAZÃO;
  \item
    Pela INTELIGÊNCIA;

    \begin{enumerate}
    \def\labelenumi{\alph{enumi}.}
    \tightlist
    \item
      Nesse caso, seria preciso admitir que não é causa de si ou
      incondicionada
    \item
      Nesse caso é CAUSADA
    \item
      Pelo RACIOCÍNIO;
    \item
      Pelo PENSAMENTO
    \end{enumerate}
  \end{itemize}
\item
  FILÓSOFOS POSTERIORES A ARISTÓTELES

  \begin{itemize}
  \tightlist
  \item
    A INTELIGÊNCIA

    \begin{itemize}
    \tightlist
    \item
      Inclina a VONTADE para certa direção;
    \item
      Não obriga nem constrange a VONTADE
    \item
      Podemos agir na direção contrária à indicada pela inteligência ou
      razão;
    \end{itemize}
  \end{itemize}
\item
  JEAN-PAUL SATRE

  \begin{itemize}
  \tightlist
  \item
    LIBERDADE

    \begin{itemize}
    \tightlist
    \item
      É a ESCOLHA INCONDICIONAL que o próprio homem faz de seu ser e de
      seu mundo.
    \item
      Estamos condenados à LIBERDADE;

      \begin{enumerate}
      \def\labelenumi{\alph{enumi}.}
      \tightlist
      \item
        É ela que define a humanidade dos humanos, sem escapatória.
      \end{enumerate}
    \end{itemize}
  \item
    Quando julgamos estar sob o PODER DE FORÇAS EXTERNAS mais poderosas
    do que nossa VONTADE, esse julgamento é uma decisão livre;

    \begin{itemize}
    \tightlist
    \item
      OUTROS outros homens, nas mesmas circunstâncias, não se curvaram
      nem se resignaram;
    \end{itemize}
  \item
    QUANDO OUTROS poderiam, nas mesmas circunstâncias, AGIR DE FORMA
    DIFERENTE a DECISÃO É LIVRE:

    \begin{itemize}
    \tightlist
    \item
      Conformar-se ou resignar-se é uma DECISÃO LIVRE;
    \item
      Afirmar-se ENFRAQUECIDO ou Afirma-se SEM FORÇAS para fazer alguma
      coisa é uma DECISÃO LIVRE;
    \end{itemize}
  \end{itemize}
\item
  Essa PRIMEIRA concepção \textbf{MANTÉM} a OPOSIÇÃO entre LIBERDADE e
  NECESSIDADE;
\end{itemize}

\hypertarget{segunda-a-concepuxe7uxe3o-que-une-necessidade-e-liberdade}{%
\paragraph{SEGUNDA: A concepção que une NECESSIDADE e
LIBERDADE}\label{segunda-a-concepuxe7uxe3o-que-une-necessidade-e-liberdade}}

\begin{itemize}
\tightlist
\item
  Apresentada:

  \begin{itemize}
  \tightlist
  \item
    Pelo ESTOICISMO no período Helenístico
  \item
    No século XVII com ESPINOSA;
  \item
    No século XIX com HEGEL;
  \end{itemize}
\item
  PRESERVA a ideia ARISTOTÉLICA que:

  \begin{itemize}
  \tightlist
  \item
    A liberdade é AUTODETERMINAÇÃO;
  \item
    É livre quem age sem ser forçado nem constrangido por nada nem por
    ninguém ( AGIR EXPONTÂNEO )
  \end{itemize}
\item
  DISTANCIA-SE da ideia ARISTOTÉLICA e de SATRE:

  \begin{itemize}
  \tightlist
  \item
    Ao NÃO SITUAR a LIBERDADE no ATO DE ESCOLHA realizado pela VONTADE
    INDIVIDUAL, \textbf{separada} da NECESSIDADE e OPOSTA A ELA;
  \end{itemize}
\item
  A LIBERDADE é colocada como parte de UM TODO NECESSÁRIO ( AGE
  LIVREMENTE quem AGE NECESSÁRIAMENTE )
\item
  Para essa perspectiva filosófica, NECESSÁRIO significa aquilo que AGE
  apenas pela FORÇA INTERNA de sua PRÓPRIA NATUREZA (Estóicos) /
  SUBSTÂNCIA (Espinosa) / ESPÍRITO (Hegel);
\item
  NATUREZA (Estóicos) / SUBSTÂNCIA (Espinosa) / ESPÍRITO (Hegel) são a
  TOTALIDADE como PODER ABSOLUTODE AÇÃO;
\item
  Como \textbf{NADA EXTERIOR} obriga a \textbf{NATUREZA}, a
  \textbf{SUBSTÂNCIA} ou o \textbf{ESPÍRITO} a AGIR, eles são LIVRES,
  pois agem apenas por seu PODER INTERNO;
\item
  Seu agir é uma NECESSIDADE LIVRE ou uma LIBERDADE NECESSÁRIA porque:

  \begin{itemize}
  \tightlist
  \item
    A NECESSIDADE não é um PODER EXTERNO que obriga a LIBERDADE a AGIR;
  \item
    A NECESSIDADE é apenas a \textbf{LEI INTERNA} que a própria
    LIBERDADE \textbf{criou para sua própria ação}
  \end{itemize}
\item
  A LIBERDADE não é um PODER INCONDICIONADO PARA ESCOLHER --- a natureza
  não escolhe, a substância não escolhe, o espírito não escolhe.
\item
  A LIBERDADE é o PODER DO TODO para \textbf{AGIR EM CONFORMIDADE
  CONSIGO MESMO}:

  \begin{itemize}
  \tightlist
  \item
    Sendo necessariamente O QUE É;
  \item
    Fazendo necessariamente O QUE FAZ;
  \item
    Sendo necessariamente O QUE É;
  \item
    Fazendo necessariamente O QUE FAZ.
  \end{itemize}
\item
  Essa SEGUNDA concepção \textbf{NÃO MANTÉM} a OPOSIÇÃO entre LIBERDADE
  e NECESSIDADE;

  \begin{itemize}
  \tightlist
  \item
    Ela afirma que a NECESSIDADE é a maneira pela qual a LIBERDADE do
    TODO se manifesta;
  \item
    A TOTALIDADE

    \begin{itemize}
    \tightlist
    \item
      É LIVRE porque:

      \begin{enumerate}
      \def\labelenumi{\alph{enumi}.}
      \tightlist
      \item
        Se põe a si mesma na existência;
      \item
        Define por si mesma as leis e as regras de sua atividade;
      \end{enumerate}
    \item
      É NECESSÁRIA porque:

      \begin{enumerate}
      \def\labelenumi{\alph{enumi}.}
      \tightlist
      \item
        Tais LEIS e REGRAS exprimem necessariamente \textbf{O QUE ELA É}
        e \textbf{O QUE ELA FAZ};
      \end{enumerate}
    \end{itemize}
  \end{itemize}
\item
  \textbf{LIBERDADE não é ESCOLHER e DELIBERAR, mas AGIR ou FAZER alguma
  coisa EM CONFORMIDADE com a NATUREZA DO AGENTE que, no caso, é o
  TODO}.
\item
  O que é a LIBERDADE HUMANA enquanto o homem é uma parte constituída
  pelo todo e que age no interior do todo?

  \begin{itemize}
  \tightlist
  \item
    São duas as respostas a essa questão:

    \begin{itemize}
    \tightlist
    \item
      A PRIMEIRA (dada pelos estoicos e por Hegel) afirma que o todo é
      racional e que suas partes também o são, sendo livres quando
      agirem em conformidade com as leis racionais do todo, para o bem
      da totalidade;
    \item
      A SEGUNDA (dada por Espinosa) afirma que as partes são de mesma
      essência que o todo e, portanto, são RACIONAIS e LIVRES como ele,
      dotadas de FORÇA INTERIOR para agir por si mesmas, de sorte que a
      LIBERDADE é TOMAR PARTE ATIVA na atividade do todo.

      \begin{enumerate}
      \def\labelenumi{\alph{enumi}.}
      \tightlist
      \item
        \textbf{TOMAR PARTE ATIVA} significa:
      \item
        Por UM LADO: a. Conhecer as CONDIÇÕES e CAUSAS estabelecidas
        pelo todo; b. Conhecer o MODO como elas determinam nossas ações;
      \end{enumerate}

      \begin{itemize}
      \tightlist
      \item
        Por OUTRO LADO (em virtude de tal conhecimento):

        \begin{enumerate}
        \def\labelenumi{\alph{enumi}.}
        \tightlist
        \item
          Não ser um joguete das CONDIÇÕES e CAUSAS que atuam sobre nós
        \item
          AGIR sobre elas também
        \end{enumerate}
      \end{itemize}
    \end{itemize}
  \end{itemize}
\item
  NÃO SOMOS LIVRES para escolher tudo;
\item
  SOMOS LIVRES para fazer tudo quanto esteja de acordo {[} Graças ao
  conhecimento que temos: (1) de nós mesmos e (2) das circunstâncias
  {]}:

  \begin{itemize}
  \tightlist
  \item
    Com nosso ser;
  \item
    Com nossa capacidade de agir;
  \end{itemize}
\item
  Para os ESTÓICOS, o homem livre é aquele

  \begin{itemize}
  \tightlist
  \item
    Cuja RAZÃO conhece:

    \begin{itemize}
    \tightlist
    \item
      A necessidade natural;
    \item
      A necessidade de sua própria natureza;
    \end{itemize}
  \item
    Tem força para guiar e dirigir a vontade para que esta exerça um
    poder absoluto sobre a irracionalidade dos instintos e impulsos,
    isto é, sobre as paixões.
  \end{itemize}
\item
  Para ESPINOSA, o homem livre é aquele que

  \begin{itemize}
  \tightlist
  \item
    AGE como CAUSA interna, completa e total de sua ação
  \item
    AGE decorrente do \textbf{desenvolvimento espontâneo} da
    \textbf{essência racional} do agente.
  \item
    Em outras palavras, assim como:

    \begin{itemize}
    \tightlist
    \item
      O todo age livremente pela necessidade de sua essência;
    \item
      O indivíduo livre age por necessidade de sua própria essência.
    \end{itemize}
  \item
    Somos livres quando realizamos nosso ser como uma potência interna
    capaz de uma pluralidade simultânea de ideias, afetos e ações que
    decorrem apenas de nosso próprio ser.
  \item
    Somos livres quando:

    \begin{itemize}
    \tightlist
    \item
      O que somos exprimem nossa FORÇA INTERNA para existir e agir
    \item
      O que sentimos exprimem nossa FORÇA INTERNA para existir e agir
    \item
      O que fazemos exprimem nossa FORÇA INTERNA para existir e agir
    \item
      O que pensamos exprimem nossa FORÇA INTERNA para existir e agir.
    \end{itemize}
  \end{itemize}
\item
  Para HEGEL, o homem livre é uma figura que aparece na história e na
  cultura sob duas formas principais:

  \begin{itemize}
  \tightlist
  \item
    Na primeira, a liberdade humana coincide com o surgimento da cultura

    \begin{itemize}
    \tightlist
    \item
      É livre o homem que não se deixa dominar pela força da natureza e
      que a vence, dobrando-a à sua vontade

      \begin{enumerate}
      \def\labelenumi{\alph{enumi}.}
      \tightlist
      \item
        Por meio do TRABALHO, da LINGUAGEM e DAS ARTES.
      \item
        Podemos notar que a LIBERDADE refere-se muito mais a uma ATITUDE
        DA HUMANIDADE, e não do INDIVÍDUO -- a uma vitória da cultura
        sobre a natureza.
      \end{enumerate}
    \end{itemize}
  \item
    Em sua outra forma, o homem livre como indivíduo livre aparece na
    história em dois momentos sucessivos:

    \begin{itemize}
    \tightlist
    \item
      O PRIMEIRO é o do surgimento do homem cristão ou o surgimento da
      interioridade cristã, que descobre a consciência como consciência
      de si;
    \item
      O SEGUNDO momento, decorrente do primeiro, é o do surgimento da
      individualidade racional moderna ou do indivíduo como consciência
      de si reflexiva
    \item
      Nesse momento, o indivíduo vê SUA RAZÃO e SUA VONTADE:

      \begin{enumerate}
      \def\labelenumi{\alph{enumi}.}
      \tightlist
      \item
        Independentes da natureza ou da necessidade natural;
      \item
        Independentes da coação de autoridades externas \textbf{na
        definição de seu pensamento e de sua vontade}.
      \end{enumerate}
    \end{itemize}
  \end{itemize}
\end{itemize}

\hypertarget{terceira-a-concepuxe7uxe3o-da-liberdade-como-possibilidade-objetiva}{%
\paragraph{TERCEIRA: A concepção da liberdade como POSSIBILIDADE
OBJETIVA}\label{terceira-a-concepuxe7uxe3o-da-liberdade-como-possibilidade-objetiva}}

\begin{itemize}
\tightlist
\item
  Essa terceira concepção busca \textbf{UNIR ELEMENTOS DAS DUAS OUTRAS};
\item
  Essa terceira concepção afirma:

  \begin{itemize}
  \tightlist
  \item
    COMO NA SEGUNDA: Que não somos um poder incondicional de escolha
    entre quaisquer possíveis, mas que nossas escolhas são condicionadas
    pelas circunstâncias em que vivemos:

    \begin{itemize}
    \tightlist
    \item
      Naturais;
    \item
      Psíquicas;
    \item
      Culturais; e
    \item
      Históricas.
    \end{itemize}
  \end{itemize}
\item
  COMO NA PRIMEIRA: Que a liberdade é um ato de DECISÃO e ESCOLHA entre
  vários possíveis.

  \begin{itemize}
  \tightlist
  \item
    Todavia, não se trata da liberdade de querer alguma coisa, e sim
    (como já dizia Espinosa) de fazer alguma coisa;
  \item
    Somos livres para fazer alguma coisa quando temos o poder de
    fazê-la.
  \end{itemize}
\item
  Essa TERCEIRA concepção da liberdade:

  \begin{itemize}
  \tightlist
  \item
    Encontramos:

    \begin{itemize}
    \tightlist
    \item
      Em pensadores marxistas (como Georg Lukács e Lucien Goldmann);
    \item
      Em pensadores vindos da fenomenologia;
    \item
      Em pensadores vindos do existencialismo (como Merleau-Ponty)
    \end{itemize}
  \item
    Introduz a NOÇÃO DE POSSIBILIDADE OBJETIVA

    \begin{itemize}
    \tightlist
    \item
      O possível não é apenas alguma coisa sentida ou percebida
      subjetivamente por nós
    \item
      O possível é também, e sobretudo, alguma coisa inscrita
      objetivamente no seio da própria necessidade

      \begin{enumerate}
      \def\labelenumi{\alph{enumi}.}
      \tightlist
      \item
        Que indica que o curso de uma situação pode ser mudado por nós,
        em certas direções e sob certas condições.
      \end{enumerate}
    \end{itemize}
  \item
    A LIBERDADE:

    \begin{itemize}
    \tightlist
    \item
      É a capacidade para perceber tais possibilidades
    \item
      É o poder para realizar aquelas ações que mudam o curso das
      coisas, dando-lhe outra direção ou outro sentido.
    \end{itemize}
  \end{itemize}
\item
  De fato:

  \begin{itemize}
  \tightlist
  \item
    EXISTIRAM FILÓSOFOS que afirmaram a LIBERDADE COMO UM PODER
    ABSOLUTAMENTE INCONDICIONAL DA VONTADE (como o fizeram, por razões
    diferentes,Kant e Sartre);
  \item
    EXISTIRAM OUTROS FILÓSOFOS que levaram em conta a TENSÃO entre nossa
    LIBERDADE e as CONDIÇÕES -- naturais, culturais, psíquicas -- que
    nos determinam.
  \end{itemize}
\item
  As discussões

  \begin{itemize}
  \tightlist
  \item
    Sobre

    \begin{itemize}
    \tightlist
    \item
      as paixões
    \item
      os interesses
    \item
      as circunstâncias histórico-sociais;
    \item
      as condições naturais;
    \end{itemize}
  \item
    Sempre estiveram presentes na ética
  \item
    Por isso, uma ideia como a de possibilidade objetiva sempre esteve
    pressuposta ou implícita nas TEORIAS SOBRE LIBERDADE.
  \end{itemize}
\end{itemize}

\hypertarget{a-morte-e-a-vida}{%
\paragraph{A morte e a Vida}\label{a-morte-e-a-vida}}

\begin{itemize}
\tightlist
\item
  Viver e morrer são a descoberta da finitude humana, de nossa
  temporalidade e de nossa identidade;
\item
  A MORTE, e somente ela, completa o que somos, dizendo o que fomos;
\item
  Por isso, os filósofos estoicos propunham que

  \begin{itemize}
  \tightlist
  \item
    Somente após a morte, quando terminam as vicissitudes da vida,
    podemos afirmar que alguém foi feliz ou infeliz.
  \item
    ``Quem não souber morrer bem terá vivido mal'', afirmou o estoico
    Sêneca
  \end{itemize}
\item
  Enquanto vivos, somos tempo e mudança, estamos sendo;
\item
  Os filósofos existencialistas disseram:

  \begin{itemize}
  \tightlist
  \item
    A existência precede a essência
  \item
    Nossa essência é a síntese do todo de nossa existência.
  \end{itemize}
\item
  Morrer é um ato solitário. Morre-se só: a essência da morte é a
  solidão.
\item
  A ética

  \begin{itemize}
  \tightlist
  \item
    É o mundo das relações intersubjetivas,isto é, entre o eu e o outro
    como sujeitos e pessoas
  \item
    É o eu e o outro como seres

    \begin{itemize}
    \tightlist
    \item
      Conscientes;
    \item
      Livres; e
    \item
      Responsáveis.
    \end{itemize}
  \end{itemize}
\item
  Nenhuma experiência evidencia tanto a dimensão essencialmente
  intersubjetiva da vida e da vida ética quanto a do DIÁLOGO.
\item
  Porque a vida é intersubjetividade corporal e psíquica e porque a vida
  ética é reciprocidade entre sujeitos;
\item
  Espinosa afirma que

  \begin{itemize}
  \tightlist
  \item
    O ser humano é mais livre na companhia dos outros do que na solidão;
  \item
    ``somente os seres humanos livres são gratos e reconhecidos uns aos
    outros'', pois os sujeitos livres são aqueles que ``nunca agem com
    fraude, mas sempre de boa-fé''.
  \end{itemize}
\end{itemize}

\hypertarget{p1---leitura-e-produuxe7uxe3o-textual}{%
\chapter{P1 - Leitura e Produção
Textual}\label{p1---leitura-e-produuxe7uxe3o-textual}}

Neste capítulo estarão contidos os resumos capítulos de livros, artigos,
monografias, dissertações, teses relacionados com a disciplina Leitura e
Produção Textual.

\hypertarget{artigo-a-importuxe2ncia-de-um-sistema-de-sauxfade-puxfablico-e-universal-no-enfrentamento-uxe0-epidemia}{%
\section{\texorpdfstring{Artigo:
\href{https://www.epsjv.fiocruz.br/noticias/reportagem/a-importancia-de-um-sistema-de-saude-publico-e-universal-no-enfrentamento-a}{A
importância de um sistema de saúde público e universal no enfrentamento
à
epidemia}}{Artigo: A importância de um sistema de saúde público e universal no enfrentamento à epidemia}}\label{artigo-a-importuxe2ncia-de-um-sistema-de-sauxfade-puxfablico-e-universal-no-enfrentamento-uxe0-epidemia}}

Resumo elaborado pelo aluno *\textbf{Daniel de Lima Claudino}, em
01/11/2022, para obtenção de nota referente ao Trabalho Acadêmico
Efetivo (TAE), conforme previsto no plano da disciplina Leitura e
Produção Textual do curso de Bacharelado em Psicologia.

\hypertarget{referuxeancia-bibliogruxe1fica}{%
\subsection{Referência
Bibliográfica}\label{referuxeancia-bibliogruxe1fica}}

GUIMARÃES, Cátia. A importância de um sistema de saúde público e
universal no enfrentamento à epidemia. Disponível em:
\url{https://www.epsjv.fiocruz.br/noticias/reportagem/a-importancia-de-um-sistema-de-saude-publico-e-universal-no-enfrentamento-a}.
Acesso em: 01 nov. 2022.

\hypertarget{resumo-informativo-analuxedtico}{%
\subsection{Resumo Informativo
(Analítico)}\label{resumo-informativo-analuxedtico}}

Cátia Guimarães inicia seu artigo afirmando que a existência de um
sistema público de saúde contribui num momento de crise sanitária como a
COVID-19. Declara ainda que esse mesmo sistema é subfinanciado e que ele
apresenta vantagens em relação ao de outros países.

Angélica Fonseca, professora-pesquisadora da Escola Politécnica de Saúde
Joaquim Venâncio (EPSJV), da Fiocruz afirma que acredita ser ingênuo
pensar que o enfrentamento da pandemia no Brasil poderia se dar fora de
um sistema público, fora do Sistema Único de Saúde (SUS). Afirma a
professora que o ``sofrimento coletivo'' não é suficiente para inverter
a lógica dos interesses particulares. Gastão Wagner, médico e professor
da Universidade Estadual de Campinas (Unicamp) destaca que se uma
situação como essa da pandemia acontecesse no Brasil antes do SUS,
``80\% a 90\% da população'' só teria como alternativa correr para o
pronto-socorro, único serviço de assistência à saúde gratuito para
qualquer pessoa naquela época, antes da criação do SUS.

O SUS foi criado com a Constituição Federal em 1988, art. 196,
declarando ser ``aúde é direito de todos e dever do Estado'' se
comprometendo ainda em garantir ``acesso universal e igualitário às
ações e serviços para sua promoção, proteção e recuperação {[} de saúde
{]}''. Essas características são importantes no momento do enfrentamento
de uma epidemia, destaca Cristiani Machado, pesquisadora e atual
vice-presidente de Ensino, Informação e Comunicação da Fiocruz.

Antes do SUS, não havia Unidades Básicas de Saúde (UBS), as pessoas não
tinham garantido o acesso hospitalar, sem contar que era muito pequeno
número de hospitais. A saúde era, principalmente a assistência médica e
hospitalar, eraconduzida pelo Ministério da Previdência e Assistência
Social (MPAS). A saúde pública era isolada da rede, sendo o Ministério
da Saúde responsável, principalmente, por ações de campanhas e vacinas,
por exemplo. O Ministério da Previdência só atendia uma parte da
população -- aquela que tinha vínculo formal de trabalho. O resto ou
pagava pelo serviço privado ou corria para o pronto socorro. Gastão
Wagner, médico e professor da Unicamp afirma que se esse modelo não
tivesse mudado, os mais de 12 milhões de desempregados e 38 milhões de
trabalhadores informais que existem hoje no Brasil {[}2020{]}
simplesmente estariam sem cobertura em meio à pandemia. O acesso da
população nesse modelo ligado ao seguro social é condicionado à inserção
no mercado de trabalho, ao status social, ao nível de renda. Esse tipo
de modelo tende a ter mais dificuldade de dar respostas integradas e
coordenadas de atenção à população e são sistemas geradores das maiores
desigualdades.

Cristiani Machado, pesquisadora da Fiocruz, destaca ainda que os países
latino-americanos, em geral, têm ``sistemas de proteção social mais
precários e sistemas de saúde mais frágeis'' do que o Brasil e ainda
estão fortemente ancorados na lógica do seguro social. ``Muitas vezes
isso atinge 50\% da população e a outra metade só tem acesso a serviços
básicos que não dão conta da maior parte dos problemas de saúde da
população'', afirma. O Brasil, na América Latina, é uma exceção no
sentido da existência do sistema único de saúde, embora, segundo
Cristiani, seja marcado por muitas contradições.

Cristiani apresenta o que ela entende ser contradições no SUS. A
existência do SUS, como sistema universal, não garante ums situação
confortável em qualquer caso. Isso é um reflexo do tamanho do desafio
único no mundo: ``Oferecer saúde pública e gratuita, entendida como um
direito, num país continental, que hoje tem quase 210 milhões de
habitantes''. A autora apresenta dados relativos à população de outros
países em relação ao Brasil e à cobertura do SUS: o Reino Unido, com
população menor que 67 milhões. No Canadá, 38 milhões. No Brasil, hoje,
temos ``162 milhões de pessoas dependem exclusivamente do SUS, sem
contar que os cerca de 47 milhões que têm planos de saúde também
utilizam o sistema público(p.~ex. como vacinação e transplantes)''.

Aliado a ela, existe também um problema crônico: o subfinanciamento,
principal obstáculo apontado por profissionais e pesquisadores da área
desde a criação do SUS. ``O SUS nunca teve recursos suficientes para a
concretização plena dos seus princípios e vem sofrendo restrições muito
importantes no período mais recente, com a Emenda Constitucional 95 e
outras medidas que estão subtraindo recursos da saúde, justamente quando
a nossa população está ficando mais idosa'', explica Cristiani. Segundo
cálculos dos economistas Francisco Funcia, Rodrigo Benevides e Carlos
Ocké-Reis, só com a Emenda Constitucional 95, que estabeleceu um teto de
gastos para o governo federal, o SUS perdeu R\$ 22,48 bilhões em entre
2018 e 2020.

Outra ``contradição'' importante do sistema brasileiro, destacada por
Cristiani, é a existência -- e o crescimento -- de um setor privado e
lucrativo da saúde, muitas vezes beneficiado por recursos públicos, por
exemplo, através de renúncia fiscal. Um gargalo do país, que precisa e
pode ser contornado a tempo, é a quantidade de leitos com terapia
intensiva disponíveis para os eventuais casos mais graves de
coronavírus.

Entre desafios e contradições, os pesquisadores não têm dúvida do saldo
positivo de se ter um sistema público e universal de saúde. Gastão
Wagner, médico e professor da Unicamp, conclui que ``Um efeito
inesperado do coronavírus é o fortalecimento dessa ideia de que a
atenção e o cuidado à saúde precisam estar fora do mercado''.

\hypertarget{artigo-psicologia-da-sauxfade-contexto-e-intervenuxe7uxe3o-da-revista-anuxe1lise-psicoluxf3gica}{%
\section{\texorpdfstring{Artigo:
\href{https://drive.google.com/file/d/1Xph9Bpk8TS42f-cGZP08vVJ3TLEqK3GZ/view?usp=sharing}{Psicologia
da Saúde: Contexto e intervenção}** da Revista \textbf{Análise
Psicológica}''}{Artigo: Psicologia da Saúde: Contexto e intervenção** da Revista Análise Psicológica''}}\label{artigo-psicologia-da-sauxfade-contexto-e-intervenuxe7uxe3o-da-revista-anuxe1lise-psicoluxf3gica}}

\hypertarget{referuxeancia-bibliografica}{%
\subsection{Referência
Bibliografica}\label{referuxeancia-bibliografica}}

CARVALHO TEIXEIRA, José A.; LEAL, Isabel. Psicologia da Saúde: Contexto
e intervenção. Revista Análise Psicológica, v. 4, n.~VIII, p.~453-458,
1990.

\hypertarget{resumo}{%
\subsection{Resumo}\label{resumo}}

\begin{enumerate}
\def\labelenumi{\arabic{enumi}.}
\tightlist
\item
  \textbf{Introdução}
\end{enumerate}

\begin{itemize}
\tightlist
\item
  A Expressão Psicologia da Saúde

  \begin{itemize}
  \tightlist
  \item
    Foi definida por Matarazzo (1980)
  \item
    CONCEITO: É a área disciplinar que diz respeito ao «papel da
    Psicologia como CIÊNCIA e como PROFISSÃO nos domínios da saúde e das
    medicinas comportamentais»
  \end{itemize}
\item
  O CONCEITO de PSICOLOGIA DA SAÚDE unificou e tomou como referências
  dois campos interdisciplinares

  \begin{itemize}
  \tightlist
  \item
    SAÚDE COMPORTAMENTAL

    \begin{itemize}
    \tightlist
    \item
      Se refere essencialmente à DIMENSÃO PREVENTIVA
    \item
      Uma subespecialidade interdisciplinar que se ocupa
      especificamente:

      \begin{itemize}
      \tightlist
      \item
        Da promoção da saúde;
      \item
        Da prevenção da doença;
      \item
        De disfunções em pessoas habitualmente saudáveis;
      \end{itemize}
    \end{itemize}
  \item
    MEDICINA COMPORTAMENTAL

    \begin{itemize}
    \tightlist
    \item
      Remete para as DIMENSÕES CURATIVA e de REABILITAÇÃO;
    \item
      Aproxima mais a Psicologia com as diversas especialidades médicas
      e cirúrgicas;
    \item
      Um campo interdisciplinar de prática clínica e de investigação que
      diz respeito a doença e a disfunções psicológicas com ela
      relacionadas.
    \end{itemize}
  \item
    CONCEITO DOS AUTORES: PSICOLOGIA DA SAÚDE

    \begin{itemize}
    \tightlist
    \item
      É a intervenção psicológica no campo da saúde e que melhor parece
      corresponder as necessidades da prática clínica.
    \item
      Permite integrar harmoniosamente os TRÊS NÍVEIS clássicos de
      PREVENÇÃO: primária, secundária e terciária
    \end{itemize}
  \end{itemize}
\item
  A abordagem psicológica da saúde e da doença é uma resposta a
  necessidade de humanização dos cuidados de saúde;
\item
  A Psicologia da Saúde é uma resposta a (1) necessidade essencial de
  INTERDISCIPLINARIEDADE da investigação científica e (2) a necessidade
  de uma COMUNICAÇÃO EFETIVA ( que alcance os RESULTADOS PRETENDIDOS )
\item
  O discurso médico e o discurso psicológico em Psicologia da Saúde.
\item
  Como será possível o diálogo interdisciplinar em Psicologia da Saúde ?
\item
  O que espera-se da Psicologia em Psicologia da Saúde ?

  \begin{itemize}
  \tightlist
  \item
    Que a Psicologia se mantenha e assegure:

    \begin{itemize}
    \tightlist
    \item
      Sua identidade própria; e
    \item
      Sua autonomia.
    \end{itemize}
  \end{itemize}
\end{itemize}

\begin{enumerate}
\def\labelenumi{\arabic{enumi}.}
\setcounter{enumi}{1}
\tightlist
\item
  \textbf{Contexto}
\end{enumerate}

\begin{itemize}
\tightlist
\item
  Psicologia da Saúde como \textbf{movimento mútuo de aproximação} entre
  a Psicologia e a Medicina
\item
  Década de 70: O modelo clássico \textbf{BIOMÉDICO} e a proposta do
  modelo \textbf{BIOPSICOSOCIAL} de Engel(1970) e Lipowski(1977)

  \begin{itemize}
  \tightlist
  \item
    \textbf{Consequências} do novo \textbf{modelo BIOPSICOSSOCIAL}:

    \begin{itemize}
    \tightlist
    \item
      Foco na \textbf{PESSOA DOENTE} ao invés de na \textbf{DOENÇA}
    \item
      Compreensão mais ampliada \textbf{do ADOECER} e \textbf{do ESTAR
      DOENTE} correlacionando com:

      \begin{itemize}
      \tightlist
      \item
        Condutas individuais;
      \item
        Personalidade;
      \item
        Estilo relacional;
      \item
        Outros aspectos psicológicos;
      \item
        Outros aspectos psicossociais.
      \end{itemize}
    \end{itemize}
  \end{itemize}
\item
  \textbf{Estilos individuais} de lidar com a \textbf{ADVERSIDADE}
\item
  Os \textbf{modelos psicológicos}: Pespectivas COMPORTAMENTAL,
  PSICANALÍTICA e SISTEMÁTICA

  \begin{itemize}
  \tightlist
  \item
    Perspectiva COMPORTAMENTAL;
  \item
    Perspectiva PSICANALÍTICA;
  \item
    Perspectiva SISTEMÁTICA.
  \end{itemize}
\item
  A progressiva \textbf{NECESSIDADE} de relacionar a Psicologia com a
  saúde física

  \begin{itemize}
  \tightlist
  \item
    Decorrente do duplo enraizamento do MODELO BIOPSICOSSOCIAL e dos
    MODELOS PSICOLÓGICOS (perspectivas COMPORTAMENTAL, PSICANALÍTICA e
    SISTEMÁTICA)
  \item
    Provocando o repensar da formação médica e da formação psicológica
  \end{itemize}
\item
  \textbf{Áreas de interesse} da Psicologia da Saúde (Subespecialidade
  da Psicologia Clínica)

  \begin{itemize}
  \tightlist
  \item
    Estudo da ADAPTAÇÃO PSICOLÓGICA e ALTERAÇÕES DO COMPORTAMENTO
    associadas:

    \begin{itemize}
    \tightlist
    \item
      Ao envelhecimento;
    \item
      À deterioração neurológica;
    \item
      À maternidade;
    \item
      Às doenças crônicas.
    \end{itemize}
  \item
    Investigação do \textbf{PAPEL} e da \textbf{INFLUÊNCIA} de fatores
    PSICOLÓGICOS e da PERSONALIDADE:

    \begin{itemize}
    \tightlist
    \item
      CAUSALIDADE MULTIFATORIAL:

      \begin{itemize}
      \tightlist
      \item
        De doenças corporais;
      \item
        Na EVOLUÇÃO, TRATAMENTO e REABILITAÇÃO de doenças corporais;
      \end{itemize}
    \end{itemize}
  \item
    Influência de VARIAVEIS PSICOLÓGICAS em ÁREAS PSICOLÓGICAS

    \begin{itemize}
    \tightlist
    \item
      Respostas individuais a vários tratamentos médicos
    \item
      Estudo de Compliance
    \item
      As RESPOSTAS PSICOLÓGICAS aos PROCEDIMENTOS CIRÚRGICOS
    \item
      O IMPACTO PSICOLÓGICO da HOSPITALIZAÇÃO
    \item
      O tratamento da DOR CRÔNICA
    \item
      Aspectos interativos do stress
    \item
      coping e adaptação
    \item
      Doenças terminais
    \end{itemize}
  \item
    Abordagem psicológica de PROMOÇÃO DA SAÚDE

    \begin{itemize}
    \tightlist
    \item
      Determinantes das mudanças de ESTILOS DE VIDA relacionados com a
      saúde;
    \end{itemize}
  \item
    Estudo dos ASPECTOS PSICOLÓGICOS associados ao

    \begin{itemize}
    \tightlist
    \item
      Stress, Tabagismo, Obesidade,Diabetes, Doenças cardiovasculares,
      Asma Brônquica e Doenças cancerosas;
    \item
      Necessidade de Avaliação;
    \item
      Necessidade de Aoio Apsicológico;
    \item
      Problemas decorrentes de novas tecnologias
    \item
      Tecnologia de Transplantes;
    \item
      Tecnologias de Reprodução;
    \end{itemize}
  \end{itemize}
\item
  O que é PSICOLOGIA DA SAÚDE

  \begin{itemize}
  \tightlist
  \item
    É um PROCESSO em constante mudança (DEVIR) que tende para superação
    das perspectivas reducionistas da Medicina ( Através de MODELOS
    INTEGRATIVOS da SAÚDE e da DOENÇA );
  \item
    É uma SUBESPECIALIDADE da PSICICOLOGIA CLÍNICA

    \begin{itemize}
    \tightlist
    \item
      Psicologia (Como CIÊNCIA e como PROFISSÃO)
    \item
      Onde a Psicologia:

      \begin{itemize}
      \tightlist
      \item
        Atua ativamente no campo da saúde e da doença;
      \item
        Dialoga produtivamente com a Medicina ( Mantendo seus MODELOS,
        DISCURSO e AUTONOMIA )
      \end{itemize}
    \item
      Onde as REALIDADES PSICOLÓGICAS tornam-se mais relevantes

      \begin{itemize}
      \tightlist
      \item
        Na PROMOÇÃO DA SAÚDE;
      \item
        Na PREVENÇÃO DA DOENÇA;
      \item
        No Adoecer;
      \item
        No Estar-doente
      \item
        Na recuperação;
      \item
        Na reabilitação conducente à reinserção familiar e comunitária
      \end{itemize}
    \end{itemize}
  \item
    CAMPOS DE INTERESSE da PSICOLOGIA DA SAÚDE

    \begin{itemize}
    \tightlist
    \item
      Estudo dos COMPORTAMENTOS DE RISCO para SAÚDE;
    \item
      Estudo dos COMPORTAMENTOS NECESSÁRIOS para MANUTENÇÃO DA SAÚDE;
    \item
      Cognições relacionadas com a saúde e com a doença;
    \item
      Aspectos psicológicos da adesão aos tratamentos;
    \item
      Aspectos psicológicos dos ambientes dos serviços de saúde;
    \item
      Estratégias de COPING relacionadas com a DOENÇA e com a
      INCAPACIDADE;
    \item
      Relações entre cuidados com a saúde e a qualidade de vida
    \item
      Aquisição precoce de COMPORTAMENTOS para a saúde;
    \item
      As condições de saúde dos técnicos de saúde;
    \end{itemize}
  \item
    ATIVIDADES da PSICOLOGIA DA SAÚDE ( Por possuir DISCURSO PRÓPRIO e
    AUTONOMIA )

    \begin{itemize}
    \tightlist
    \item
      Prevenção de saúde;
    \item
      Avaliação de saúde;
    \item
      Apoio Psicológico ligado aos cuidados de saúde;
    \item
      Investigações ligadas aos cuidados de saúde;
    \end{itemize}
  \end{itemize}
\item
  Dada a COMPLEXIDADE DE QUESTÕES que a Psicologia tem que tratar,
  existe DIVERSIDADE:

  \begin{itemize}
  \tightlist
  \item
    De modelos de formação;
  \item
    De modelos de informação;
  \item
    De contribuições teóricas
  \item
    De modelos de intervençaõ;
  \end{itemize}
\item
  COMO TRABALHA o PSICÓLOGO CLÍNICO com Psicologia da Saúde:

  \begin{itemize}
  \tightlist
  \item
    APLICANDO

    \begin{itemize}
    \tightlist
    \item
      Teorias básicas
    \item
      Metodologias de avaliação
    \item
      Metodologias de investigação
    \item
      Modelos de intervenção da Psicologia no campo da saúde
    \end{itemize}
  \item
    Numa PERSPECTIVA DE ABORDAGEM: Globalizante
  \end{itemize}
\item
  Funcionamento BIOLÓGICO x funcionamento PSICOLÓGICO (Roessler \&
  Decker, 1986).

  \begin{itemize}
  \tightlist
  \item
    DISFUNÇÕES BIOLÓGICAS podem produzir reações adversas ao
    FUNCIONAMENTO PSICOLÓGICO
  \item
    MUDANÇAS PSICOLÓGICAS e SOCIAIS podem produzir alterações no
    FUNCIONAMENTO BIOLÓGICO;
  \end{itemize}
\end{itemize}

\begin{enumerate}
\def\labelenumi{\arabic{enumi}.}
\setcounter{enumi}{2}
\tightlist
\item
  \textbf{Intervenção}
\end{enumerate}

\begin{itemize}
\tightlist
\item
  O OBJETO da Psicologia da Saúde ( definição ainda IMPRECISA )

  \begin{itemize}
  \tightlist
  \item
    É a EXPERIÊNCIA PSICOLÓGICA
  \item
    É a RELAÇÃO que os sujeitos estabelecem:

    \begin{itemize}
    \tightlist
    \item
      Com seu estado de saúde ou de doença;
    \item
      Com acontecimentos que associam-se frequentemente a MOMENTOS DE
      CRISE:

      \begin{itemize}
      \tightlist
      \item
        Gravidez;
      \item
        Puberdade;
      \item
        Envelhecimento;
      \item
        Menopausa;
      \end{itemize}
    \item
      Com especificidades biológicas
    \end{itemize}
  \end{itemize}
\item
  A PSICOLOGIA DA SAÚDE trabalha com vivências que o sujeito

  \begin{itemize}
  \tightlist
  \item
    Experimenta
  \item
    Projeta
  \item
    Reativa
  \end{itemize}
\item
  Para PSICOLOGIA DA SAÚDE o que INTERESSA são:

  \begin{itemize}
  \tightlist
  \item
    As FORMAS pelas quais o sujeito lida com os acontecimentos
  \item
    Os MOTIVOS pelos quais o sujeito lida com os acontecimentos
  \end{itemize}
\item
  O CONCEITO DE SAÚDE

  \begin{itemize}
  \tightlist
  \item
    Bem-estar físico, psicológico e social:

    \begin{itemize}
    \tightlist
    \item
      Do sujeito
    \item
      Do grupo
    \item
      Da comunidade
    \end{itemize}
  \item
    É mais que a AUSÊNCIA DE SINTOMAS
  \item
    É mais que DESVIOS em relação à MÉDIA
  \item
    É um sentir-se bem INDIVIDUALIZADO e SUBJETIVO
  \item
    TRADUZ uma representação social da nossa época
  \item
    É centrada no HOMEM e não na patologia ou entidades nosológicas;
  \end{itemize}
\item
  Os OBJETIVOS e METODOLOGIAS da Psicologia da Saúde

  \begin{itemize}
  \tightlist
  \item
    São os objetivos específicos da Psicologia Geral
  \item
    São os específicos da Psicologias Clínica
  \item
    São centrados sobre um terreno que recebe da Medicina as suas
    CATEGORIAS DE INTERVENÇÃO
  \end{itemize}
\item
  OS OBJETIVOS da PSICOLOGIA DA SAÚDE são: ( O que a Psicologia da Saúde
  pretende ? Em resumo, que o sujeito possa lidar o melhor possível com
  a nova situação )

  \begin{itemize}
  \tightlist
  \item
    Optimização dos RECURSOS AFETIVOS e COGNITIVOS do sujeito
  \item
    Adoção de ESTRATÉGIAS adequadas para SUPERAÇÃO DE CRISES
  \item
    REFORÇO de DEFESAS eventualmente enfraquecidas
  \end{itemize}
\item
  Exemplos de Atividades de INTERVENÇÃO em Psicologia da Saúde

  \begin{itemize}
  \tightlist
  \item
    Prevenção, avaliação, tratamento e reabilitação de disfunções
    psicológicas em doentes físicos;
  \item
    Aconselhamento e troca de informações com os outros técnicos de
    saúde sobre aspectos psicológicos (e psicossociais) dedoentes
    físicos, particularmente em estudos de casos individuais;
  \item
    Tratamento psicológico de reacções psicológicas as doenças físicas,
    com incidência particular na experiência vivida da doença e das suas
    limitações;
  \item
    Prevenção e/ou tratamento de condutas desajustadas que são
    consequência ou mesmo parte integrante das doenças corporais;
  \item
    Identificação e suporte de sujeitos em risco psicológico colocados
    perante o adoecer corporal;
  \item
    Aconselhamento e suporte de problemas familiares relacionados com a
    doença física de um dos seus membros;
  \item
    Facilitação da comunicação entre as pessoas doentes e as equipas de
    saúde, prevenindo conflitos potenciais e manejando os manifestos;
  \item
    Participação activa em programas de reabilitação de doentes físicos,
    particularmente com doenças crónicas invalidantes e exigindo
    processo activo de reabilitação psicológica e psicossocial;
  \item
    Investigação nas áreas de intervenção dos factores psicológicos nas
    doenças corporais e, ainda, na organização e funcionamento dos
    serviços de saúde e seu impacto psicológico sobre as pessoas
    doentes;
  \item
    Estudo e implementação da modificação de comportamentos e de estilos
    de vida, necessária para a conservação da saúde e prevenção da
    doença.\\
  \end{itemize}
\item
  Modelo psicológicos de INFORMAÇÃO, FORMAÇÃO e INTERVENÇÃO em
  Psicologia da Saúde

  \begin{itemize}
  \tightlist
  \item
    Não existe um modelo único, existem vários;
  \item
    Usam-se vários modelos de acordo com as NECESSIDADES e
    CIRCUNSTÂNICAS de cada caso individual, sempre com a necessária
    FLEXIBILIDADE indispensável a prática clínica;
  \end{itemize}
\item
  Os INSTRUMENTOS FUNDAMENTAIS de INTERVENÇÃO ( Específicos da Clínica
  Psicológica )

  \begin{itemize}
  \tightlist
  \item
    Entrevista clínica
  \item
    Exame Psicológico
  \item
    Psicoterapia
  \end{itemize}
\item
  A importância da PSICOTERAPIA em Psicologia da Saúde

  \begin{itemize}
  \tightlist
  \item
    O que habitualmente se pede ao Psicólogo Clínico:

    \begin{itemize}
    \tightlist
    \item
      Que colabore na CLARIFICAÇÃO de uma situação
    \item
      Que ajude o sujeito a MUDAR, PROMOVENDO um melhor ajustamento
      psicológico:

      \begin{itemize}
      \tightlist
      \item
        À situação
      \item
        Às consequências da situação
      \end{itemize}
    \end{itemize}
  \end{itemize}
\item
  Definição de PSICOTERAPIA DE APOIO

  \begin{itemize}
  \tightlist
  \item
    Uma forma de tratamento psicológico para sujeitos com problemas
    físicos (ou mentais crónicos) para os quais a MUDANÇA RADICAL não
    constitui um OBJETIVO REALISTA (Sidney Bloch,1979).
  \item
    Ela não inviabiliza:

    \begin{itemize}
    \tightlist
    \item
      Outra alternativa psocoterapêutica ulterior;
    \item
      Qualquer encaminhamento considerado adequado ou possível
    \end{itemize}
  \item
    Os OBJETIVOS da PSICOTERAPIA DE APOIO

    \begin{itemize}
    \tightlist
    \item
      Promover o melhor funcionamento psicológico possível, reforçando
      as capacidades do sujeito para lidar com os vários aspectos da sua
      vida e com a adversidade;
    \item
      Aumentar a auto-estima e tornar a pessoa cada vez mais consciente
      da realidade;
    \item
      Prevenir as eventuais recidivas, combater a dependência e outros
      factores que possam contribuir para o aparecimento de cronicidade
      psicológica;
    \item
      Vir a transferir a fonte de apoio (pelo menos em parte) para a
      família e rede social de apoio.
    \end{itemize}
  \end{itemize}
\item
  Os COMPONENTES PRINCIPAIS da INTERVENÇÃO terapêutica

  \begin{itemize}
  \tightlist
  \item
    Transmissão de segurança;
  \item
    Explicação;
  \item
    Sugestão;
  \item
    Aconselhamento;
  \item
    Encorajamento;
  \item
    Modificação de circunstâncias ambienciais;
  \item
    Permissão para a exteriorização emocional e afectiva
  \item
    Uso de técnicas de intervenção cognitiva e comportamental diversas,
    consoante as especificidades do sujeito e da situação que podem ser
    ÚTEIS EM SITUAÇÕES ESPECÍFICAS:

    \begin{itemize}
    \tightlist
    \item
      Tratamento da dor crónica;
    \item
      Mudança de comportamentos de risco em doentes coronários;
    \item
      Preparação psicológica para a cirurgia;
    \item
      Tabagismo, etc.
    \end{itemize}
  \end{itemize}
\item
  A Psicologia da Saúde é um reencontro entre a CONGIÇÃO e o AFETO.
\item
  \textbf{OBSERVAÇÃO}:

  \begin{itemize}
  \tightlist
  \item
    Definição de COPING: ``tentativa ou empenho para lidar com
    exigências externas (do ambiente) ou internas (do próprio sujeito)
    percebidas como sobrecarregando ou excedendo os recursos da
    pessoa.''(Folkman e Lazarus)''
    \href{https://www.psicologia.pt/artigos/ver_opiniao.php?codigo=AOP0216}{Site
    \textbf{Psicologia.pt}}
  \end{itemize}
\end{itemize}

\hypertarget{resenha}{%
\subsection{Resenha}\label{resenha}}

Inicialmente, os autores estabelecem os necessários esclarecimentos
terminológicos para o desenvolvimento adequado do tema do artigo. Eles
afirmam que a expressão Psicologia da Saúde foi definida por Matarazzo
(1980). Afirmam que ela é a área disciplinar que diz respeito ao ``papel
da Psicologia como CIÊNCIA e como PROFISSÃO nos domínios da saúde e das
medicinas comportamentais''. O conceito de psicologia da saúde unificou
e tomou como referências dois campos interdisciplinares SAÚDE
COMPORTAMENTAL e MEDICINA COMPORTAMENTAL. Para os autores, a Psicologia
da Saúde é a intervenção psicológica no campo da saúde e que melhor
parece corresponder as necessidades da prática clínica.

A abordagem psicológica da saúde e da doença é uma resposta a
necessidade de humanização dos cuidados de saúde e é uma resposta a
necessidade essencial de INTERDISCIPLINARIEDADE da investigação
científica, assim como a necessidade de uma COMUNICAÇÃO EFETIVA para que
alcance os RESULTADOS PRETENDIDOS no âmbito dos serviços de saúde.

A autora demonstra de forma clara e objetiva que o surgimento dessa
subárea da Psicologia Clínica aproximou o discurso médico e o discurso
psicológico em Psicologia da Saúde. Essa aproximação torna-se possivel
com o diálogo interdisciplinar da Psicologia da Saúde com diversas
outras áreas da saúde. Para cumprir esse desiderato (humanização e
comunicação efetiva), espera-se que a Psicologia (em Psicologia da
Saúde) mantenha e assegure sua identidade própria e sua autonomia.

Em seguida a autora esclarece como surgiu a Psicologia da Saúde como
\textbf{movimento mútuo de aproximação} entre a Psicologia e a Medicina.
Na década de 70, o modelo clássico era o \textbf{BIOMÉDICO}. A partir
dessa década surge a proposta do modelo \textbf{BIOPSICOSOCIAL} de
Engel(1970) e Lipowski(1977). O foco passa a ser a pessoa doente ao
invés da doença. Isso demonstra com claridade haver o progresso de uma
compreensão mais ampliada \textbf{do ADOECER} e \textbf{do ESTAR
DOENTE}, correlacionando condutas individuais, personalidade, estilo
relacional, outros aspectos psicológicos entre outros aspectos
psicossociais. Consideração ainda, estilos individuais de lidar com a
\textbf{adversidade}.

A progressiva \textbf{NECESSIDADE} de relacionar a Psicologia com a
saúde física decorrente do duplo enraizamento do MODELO BIOPSICOSSOCIAL
e dos MODELOS PSICOLÓGICOS (perspectivas COMPORTAMENTAL, PSICANALÍTICA e
SISTEMÁTICA) provocou um repensar da formação médica e da formação
psicológica.

As \textbf{áreas de interesse} da Psicologia da Saúde (Subespecialidade
da Psicologia Clínica) são bastante variadas, mas sempre carregam a
perspectiva multidisciplinar de modelo BIOPSICOSOCIAL, estudando a
\textbf{adaptação psicológica} e as \textbf{alterações do comportamento}
associadas. Como exemplo, sitam-se o envelhecimento, a deterioração
neurológica, a maternidade, doença e doenças crônicas. São temas de
subespecialidade a investigação do \textbf{papel} e da
\textbf{influência} de fatores PSICOLÓGICOS e da PERSONALIDE a
causalidade multifatorial de doenças corporais na evolução, tratamento e
reabilitação de doenças corporais. Estuda-se a influência de variáveis
psicológicas em respostas individuais a vários tratamentos médicos,
estudo de compliance, respostas psicológicas aos procedimentos
cirúrgicos, o impacto psicológico da hospitalização, o tratamento da dor
crônica, aspectos interativos do stress, coping e adaptação e doenças
terminais. Trata-se ainda da abordagem psicológica de \textbf{promoção
da saúde} dos determinantes das mudanças de ESTILOS DE VIDA relacionados
com a saúde, estudo dos ASPECTOS PSICOLÓGICOS associados ao stress,
tabagismo, obesidade, diabetes, doenças cardiovasculares, asma brônquica
e doenças cancerosas.

Os autores demonstram que a Psicologia Saúde é um PROCESSO em constante
mudança (DEVIR) que tende para superação das perspectivas reducionistas
da Medicina ( Através de MODELOS INTEGRATIVOS da SAÚDE e da DOENÇA );

Os \textbf{campos de interesse} da Psicologia da Saúde são diversos,
estudando os comportamentos de risco para a saúde, os comportamentos
necessários para manutenção da saúde, as cognições relacionadas com a
saúde e com a doença, os aspectos psicológicos da adesão aos
tratamentos, os aspectos psicológicos dos ambientes dos serviços de
saúde, as estratégias de COPING relacionadas com a doença e com a
incapacidade, relações entre cuidados com a saúde e a qualidade de vida,
aquisição precoce de comportamentos para a saúde, entre outros.

A Psicologia da Saúde apresenta discurso próprio e autonomia, atuando na
prevenção de saúde, avaliação de saúde, apoio psicológico ligado aos
cuidados de saúde e investigações ligadas aos cuidados de saúde;

Diante desse contexto complexo de atuação, a Psicologia da Saúde possui
uma diversidade de modelos de formação, de modelos de informação, de
contribuições teóricas e de modelos de intervenção. O modelo
biopsicosocial afirma que o funcionamento biológico pode afetar o
funcionamento psicológico, assim comomudanças psicológicas e sociais
podem produzir alterações no funcionamento biológico.

\hypertarget{p1---metodologia-cientuxedfica}{%
\chapter{P1 - Metodologia
Científica}\label{p1---metodologia-cientuxedfica}}

capa-livro-fundamentos-metodologia-cientifica-lakatos.png

\hypertarget{resumo-do-livro-fundamentos-de-metodologia-cientuxedfica-marconi-lakatos-teixeira-2017}{%
\section{\texorpdfstring{Resumo do Livro ``\textbf{Fundamentos de
Metodologia Científica}'' (MARCONI; LAKATOS; TEIXEIRA,
2017)}{Resumo do Livro ``Fundamentos de Metodologia Científica'' (MARCONI; LAKATOS; TEIXEIRA, 2017)}}\label{resumo-do-livro-fundamentos-de-metodologia-cientuxedfica-marconi-lakatos-teixeira-2017}}

Figura - Livro ``Fundamentos de Metodologia Científica'' (MARCONI;
LAKATOS; TEIXEIRA, 2017)

\hypertarget{capuxedtulo-8---pesquisa}{%
\subsection{Capítulo 8 - Pesquisa}\label{capuxedtulo-8---pesquisa}}

\hypertarget{conceito-de-pesquisa}{%
\subsubsection{Conceito de Pesquisa}\label{conceito-de-pesquisa}}

\begin{itemize}
\tightlist
\item
  \textbf{Bagno} (2010, p.~17 apud MARCONI; LAKATOS, p.~185) afirma que
  a palavra PESQUISA chegou até o Português através do Espanhol, que por
  sua vez assimilou do Latim:

  \begin{itemize}
  \tightlist
  \item
    Latim \emph{PERQUIRO}:

    \begin{itemize}
    \tightlist
    \item
      \textbf{Significado}: 1. Procurar por toda parte; buscar com
      cuidado. 2.Inquirir. 3. Perguntar; indagar bem, aprofundar na
      busca
    \item
      O significado dessa palavra em Latim insiste na \textbf{ideia de
      uma busca feita com cuidado e profundidade}.
    \end{itemize}
  \end{itemize}
\item
  \textbf{Ander-Egg} (1978, p.~28 apud MARCONI; LAKATOS, p.~185) a
  PESQUISA é:

  \begin{itemize}
  \tightlist
  \item
    Um PROCEDIMENTO:

    \begin{itemize}
    \tightlist
    \item
      Formal
    \item
      Reflexivo ( Método de Pensamento );
    \item
      Sistemático;
    \item
      Controlado;
    \item
      Crítico
    \end{itemize}
  \item
    Que PERMITE, em qualquer campo do conhecimento, DESCOBRIR

    \begin{itemize}
    \tightlist
    \item
      Novos fatos ou dados;
    \item
      Relações ou leis
    \end{itemize}
  \item
    Que REQUER

    \begin{itemize}
    \tightlist
    \item
      Tratamento científico
    \end{itemize}
  \item
    Que CONSTITUI

    \begin{itemize}
    \tightlist
    \item
      Caminho para CONHECER A REALIDADE
    \item
      Caminho para CONHECER VERDADES PARCIAIS
    \end{itemize}
  \end{itemize}
\end{itemize}

\hypertarget{passos-para-desenvolvimento-de-um-projeto-de-pesquisa}{%
\subsubsection{Passos para DESENVOLVIMENTO DE UM PROJETO DE
PESQUISA}\label{passos-para-desenvolvimento-de-um-projeto-de-pesquisa}}

\begin{enumerate}
\def\labelenumi{\arabic{enumi}.}
\tightlist
\item
  \textbf{Seleção} do tópico (Problema) para investigação;
\item
  \textbf{Definição} e \textbf{diferenciação} do problema;
\item
  \textbf{Levantamento} de hipóteses de trabalho;
\item
  \textbf{Coleta}, \textbf{sistematização} e \textbf{classificação} dos
  dados;
\item
  \textbf{Análise} e \textbf{interpretação} dos dados;
\item
  \textbf{Elaboração} do relatório do resultado da pesquisa
\end{enumerate}

\hypertarget{planejamento-da-pesquisa}{%
\paragraph{Planejamento da Pesquisa}\label{planejamento-da-pesquisa}}

\hypertarget{preparauxe7uxe3o-da-pesquisa}{%
\paragraph{Preparação da Pesquisa}\label{preparauxe7uxe3o-da-pesquisa}}

\begin{enumerate}
\def\labelenumi{\arabic{enumi}.}
\tightlist
\item
  Decisão;
\item
  Especificação dos objetivos;
\item
  Elaboração de um plano de trabalho;
\item
  Constituição da equipe de trabalho;
\item
  Levantamento de recursos e cronograma
\end{enumerate}

\hypertarget{fases-da-pesquisa}{%
\subsubsection{Fases da Pesquisa}\label{fases-da-pesquisa}}

\begin{enumerate}
\def\labelenumi{\arabic{enumi}.}
\tightlist
\item
  Escolha do tema;
\item
  Levantamento de dados;
\item
  Formulação do problema;
\item
  Definição dos termos;
\item
  Construção de hipóteses;
\item
  Indicação de variáveis;
\item
  Delimitação da pesquisa;
\item
  Amostragem;
\item
  Seleção de métodos e técnicas;
\item
  Organização do instrumental de pesquisa;
\item
  Teste de instrumentos e procedimentos
\end{enumerate}

\hypertarget{execuuxe7uxe3o-da-pesquisa}{%
\subsubsection{Execução da Pesquisa}\label{execuuxe7uxe3o-da-pesquisa}}

\begin{enumerate}
\def\labelenumi{\arabic{enumi}.}
\tightlist
\item
  C3oleta de dados;
\item
  Elaboração dos dados;
\item
  Análise e interpretação dos dados;
\item
  Representação dos dados;
\item
  Conclusões
\end{enumerate}

\hypertarget{relatuxf3rio-de-pesquisa}{%
\subsubsection{Relatório de Pesquisa}\label{relatuxf3rio-de-pesquisa}}

\hypertarget{planejamento-da-pesquisa-preparauxe7uxe3o-da-pesquisa}{%
\subsubsection{PLANEJAMENTO DA PESQUISA: Preparação da
Pesquisa}\label{planejamento-da-pesquisa-preparauxe7uxe3o-da-pesquisa}}

\hypertarget{decisuxe3o}{%
\paragraph{Decisão}\label{decisuxe3o}}

\begin{itemize}
\tightlist
\item
  Esse é o momento em que o pesquisador \textbf{toma a decisão} de
  \textbf{realizar ou não realizar} a pesquisa;
\item
  A pesquisa pode ser realizada \textbf{no INTERESSE}:

  \begin{itemize}
  \tightlist
  \item
    Do Próprio Pesquisador;
  \item
    De Alguém
  \item
    De Alguma Entidade;
  \end{itemize}
\item
  Nem sempre é fácil se dereminar \textbf{o que se pretente investigar}
\item
  A INVESTIGAÇÃO PRESSUPÕE:

  \begin{itemize}
  \tightlist
  \item
    Uma série de CONHECIMENTOS ANTERIORES;
  \item
    Metodologia adequada
  \end{itemize}
\end{itemize}

\hypertarget{especificauxe7uxe3o-dos-objetivos}{%
\paragraph{Especificação dos
objetivos}\label{especificauxe7uxe3o-dos-objetivos}}

\begin{itemize}
\tightlist
\item
  O OBJETIVO torna \textbf{EXPLÍCITO O PROBLEMA};
\item
  Os OBJETIVOS orientam a pesquisa para se \textbf{SABER}:

  \begin{itemize}
  \tightlist
  \item
    O QUE se vai PROCURAR;
  \item
    O QUE se pretende ALCANÇAR
  \end{itemize}
\item
  Os objetivos devem \textbf{SER}

  \begin{itemize}
  \tightlist
  \item
    LIMITADOS
  \item
    CLARAMENTE DEFINIDOS
  \end{itemize}
\item
  Os objetivos podem \textbf{DEFINIR}

  \begin{itemize}
  \tightlist
  \item
    A natureza do trabalho;
  \item
    O tipo do problema;
  \item
    O material a coletar
  \end{itemize}
\end{itemize}

\begin{Shaded}
\begin{Highlighting}[]
\NormalTok{flowchart LR}
\NormalTok{A(Os \textless{}b\textgreater{}OBJETIVOS\textless{}/b\textgreater{}){-}{-}\textgreater{}B(Podem SER)}
\NormalTok{A{-}{-}\textgreater{}C(Respondem as PERGUNTAS)}
\NormalTok{C{-}{-}\textgreater{}R(Por Quê?)}
\NormalTok{C{-}{-}\textgreater{}S(Para Quê)}
\NormalTok{C{-}{-}\textgreater{}T(Para Quem?)}
\NormalTok{B{-}{-}\textgreater{}D((a))}
\NormalTok{D{-}{-}\textgreater{}H(Intrínsecos)}
\NormalTok{D{-}{-}\textgreater{}M(Extrínsecos)}
\NormalTok{B{-}{-}\textgreater{}E((b))}
\NormalTok{E{-}{-}\textgreater{}I(Teóricos)}
\NormalTok{E{-}{-}\textgreater{}N(Práticos)}
\NormalTok{B{-}{-}\textgreater{}F((c))}
\NormalTok{F{-}{-}\textgreater{}J(Gerais)}
\NormalTok{F{-}{-}\textgreater{}O(Específicos)}
\NormalTok{B{-}{-}\textgreater{}G((d))}
\NormalTok{G{-}{-}\textgreater{}K(De CURTO Prazo)}
\NormalTok{G{-}{-}\textgreater{}P(De LONGO Prazo)}
\end{Highlighting}
\end{Shaded}

\hypertarget{elaborauxe7uxe3o-de-um-plano-de-trabalho}{%
\paragraph{Elaboração de um plano de
trabalho}\label{elaborauxe7uxe3o-de-um-plano-de-trabalho}}

\begin{itemize}
\tightlist
\item
  É um passo a passo de como o trabalho de pesquisa será realizado que
  facilita ou aumenta as suas chances de viabilidade;
\item
  Imprime ordem lógica a sua execução;
\item
  Tornam claros os RECURSOS \textbf{materiais}, \textbf{humanos} e de
  \textbf{tempo} necessários a sua execução;
\item
  \textbf{Pode} ou \textbf{não} ser alterado durante sua execução;
\end{itemize}

\hypertarget{constituiuxe7uxe3o-da-equipe-de-trabalho}{%
\paragraph{Constituição da equipe de
trabalho}\label{constituiuxe7uxe3o-da-equipe-de-trabalho}}

\begin{itemize}
\tightlist
\item
  Engloba

  \begin{itemize}
  \tightlist
  \item
    Recrutamento de pessoal;
  \item
    Treinamento de pessoal;
  \item
    Distribuição de tarefas;
  \item
    Indicação dos locais de trabalho;
  \item
    Indicação dos materiais e equipamentos
  \end{itemize}
\item
  \textbf{Observação}: A pesquisa também pode ser realizada por
  \textbf{uma pessoa} apenas.
\end{itemize}

\hypertarget{levantamento-de-recursos-e-cronograma}{%
\paragraph{Levantamento de recursos e
cronograma}\label{levantamento-de-recursos-e-cronograma}}

\begin{itemize}
\tightlist
\item
  O pesquisados deve fazer DE ANTEMÃO a previsão dos gastos necessários
  para a realização da pesquisa ( orçamento o mais aproximado possível
  do montante de recursos );
\item
  Não pode falta um cronograma para executar a pesquisa em cada uma de
  suas etapas
\end{itemize}

\begin{Shaded}
\begin{Highlighting}[]
\NormalTok{flowchart LR}
\NormalTok{A(O \textless{}b\textgreater{}CRONOGRAMA\textless{}/b\textgreater{} e\textless{}br\textgreater{}Os \textless{}b\textgreater{}RECURSOS\textless{}/b\textgreater{}){-}{-}\textgreater{}C(Respondem as PERGUNTAS)}
\NormalTok{C{-}{-}\textgreater{}R(Quando ?)}
\NormalTok{C{-}{-}\textgreater{}S(Quanto ?)}
\end{Highlighting}
\end{Shaded}

\hypertarget{planejamento-da-pesquisa-fases-da-pesquisa}{%
\subsubsection{PLANEJAMENTO DA PESQUISA: Fases da
Pesquisa}\label{planejamento-da-pesquisa-fases-da-pesquisa}}

\hypertarget{escolha-do-tema}{%
\paragraph{Escolha do tema;}\label{escolha-do-tema}}

\begin{itemize}
\tightlist
\item
  O \textbf{TEMA} é o \textbf{ASSUNTO} que se deseja estudar e
  pesquisar;
\item
  A \textbf{DEFINIÇÃO DO TEMA} pode perdurar por toda a pesquisa e
  deverá ser \textbf{FREQUENTEMENTE REVISTO};
\item
  ESCOLHA um TEMA/ASSUNTO

  \begin{itemize}
  \tightlist
  \item
    Que MEREÇA ser investigado cientificamente;
  \item
    Que TENHA CONDIÇÕES de ser, em função da pesquisa:

    \begin{itemize}
    \tightlist
    \item
      Formulado;
    \item
      Delimitado
    \item
      Exequível (disponibilidade de tempo do pesquisador, background
      acadêmico, recursos, etc\ldots)
    \end{itemize}
  \end{itemize}
\item
  O TEMA/ASSUNTO deve SER:

  \begin{itemize}
  \tightlist
  \item
    Preciso
  \item
    Bem determinado
  \item
    Específico
  \end{itemize}
\end{itemize}

\begin{Shaded}
\begin{Highlighting}[]
\NormalTok{flowchart LR}
\NormalTok{A(A \textless{}b\textgreater{}escolha do TEMA\textless{}/b\textgreater{}){-}{-}\textgreater{}B(Responde a PERGUNTA){-}{-}\textgreater{}C(\textless{}b\textgreater{}O QUE\textless{}/b\textgreater{} será\textless{}br\textgreater{}PESQUISADO/EXPLORADO ?)}
\end{Highlighting}
\end{Shaded}

\hypertarget{levantamento-de-dados}{%
\paragraph{Levantamento de dados;}\label{levantamento-de-dados}}

\begin{itemize}
\tightlist
\item
  São TRÊS as formas de OBTENÇÃO DE DADOS:

  \begin{itemize}
  \tightlist
  \item
    Pesquisa Bibliográfica
  \item
    Pesquisa Documental
  \item
    Contatos Diretos \#\#\#\#\# Pesquisa Bibliográfica
  \end{itemize}
\item
  Apanhado geral sobre os \textbf{principais trabalhos realizados}
\item
  Procura-se, relacionados ao TEMA:

  \begin{itemize}
  \tightlist
  \item
    Dados atuais;
  \item
    Dados relevantes;
  \item
    Indícios importantes
  \item
    Subsídios importantes
  \end{itemize}
\item
  Pode orientar indagações sore a pesquisa
\item
  ANTES de qualquer PESQUISA DE CAMPO deve-se realizar uma ANÁLISE
  MINUCIOSA de FONTES DOCUMENTAIS que sirvam de suporte à pesquisa
\end{itemize}

\hypertarget{investigauxe7uxe3o-preliminar}{%
\paragraph{Investigação
Preliminar}\label{investigauxe7uxe3o-preliminar}}

\begin{itemize}
\tightlist
\item
  Também chamada de \textbf{ESTUDO EXPLORATÓRIO}.
\item
  Deve ser realizada através de \textbf{DOIS ASPECTOS}:

  \begin{itemize}
  \tightlist
  \item
    \textbf{DOCUMENTOS}

    \begin{itemize}
    \tightlist
    \item
      \textbf{PRINCIPAIS TIPOS DE DOCUMENTOS}

      \begin{enumerate}
      \def\labelenumi{\alph{enumi}.}
      \tightlist
      \item
        Fontes \textbf{PRIMÁRIAS}:
      \item
        Dados Históricos;
      \item
        Dados Bibliográficos;
      \item
        Dados Estatísticos;
      \item
        Documentação \textbf{pessoal}; a. Autobiografias b. Certidão de
        nascimento c.~Certidão de óbito d.~Diários e. Memórias
      \item
        Registros de natureza pública/privada
      \item
        Fontes \textbf{SECUNDÁRIOS}
      \item
        Imprensa em geral
      \item
        Obras literárias
      \end{enumerate}
    \end{itemize}
  \item
    \textbf{CONTATOS DIRETOS / PESQUISA DE CAMPO / PESQUISA DE
    LABORATÓRIO}

    \begin{itemize}
    \tightlist
    \item
      São realizados com PESSOAS que podem \textbf{fornecer os dados
      diretamente};
    \item
      São realizadas com PESSOAS que podem \textbf{sugerir fontes de
      informações úteis};
    \end{itemize}
  \end{itemize}
\end{itemize}

\hypertarget{formulauxe7uxe3o-do-problema}{%
\paragraph{Formulação do problema;}\label{formulauxe7uxe3o-do-problema}}

\begin{itemize}
\tightlist
\item
  PROBLEMA:

  \begin{itemize}
  \tightlist
  \item
    É uma DIFICULDADE, teórica ou prática, NO CONHECIMENTO de alguma
    coisa;
  \item
    É uma DIFICULDADE para a qual SE QUER ENCONTRAR uma solução;
  \end{itemize}
\end{itemize}

\hypertarget{definiuxe7uxe3o-do-problema}{%
\paragraph{Definição do Problema}\label{definiuxe7uxe3o-do-problema}}

\begin{itemize}
\tightlist
\item
  O que é \textbf{DEFINIR O PROBLEMA} ?

  \begin{itemize}
  \tightlist
  \item
    É especificar a \textbf{DIFICULDADE} em detalhes precisos e exatos;
  \item
    É especificar a \textbf{DIFICULDADE} com clareza, concisão e
    objetividade
  \end{itemize}
\item
  A \textbf{CLAREZA DO PROBLEMA} ajuda na CONTRUÇÃO DA HIPÓTESE CENTRAL;
\item
  \textbf{COMO DEVE SER DEFINIDO} o \textbf{PROBLEMA} ?

  \begin{itemize}
  \tightlist
  \item
    Na forma INTERROGATIVA
  \item
    Delimitando-o de forma a indicar as \textbf{VARIÁVEIS} que
    \textbf{AFETAM O ESTUDO} com possíveis \textbf{RELAÇÕES ENTRE SI}.
  \end{itemize}
\item
  A \textbf{FORMULAÇÃO DE UM PROBLEMA}

  \begin{itemize}
  \tightlist
  \item
    Requer \textbf{CONHECIMENTOS PRÉVIOS} no assunto, exigindo:

    \begin{itemize}
    \tightlist
    \item
      Materiais informativos
    \item
      Imaginação criadora
    \end{itemize}
  \end{itemize}
\item
  Deve-se evitar PROBLEMAS MUITO ABRANGENTES, pois tornam a pesquisa
  MAIS COMPLEXA;
\end{itemize}

\hypertarget{apuxf3s-a-definiuxe7uxe3o-do-problema}{%
\paragraph{APÓS a Definição do
Problema}\label{apuxf3s-a-definiuxe7uxe3o-do-problema}}

\begin{itemize}
\tightlist
\item
  \textbf{APÓS FORMULADO UM PROBLEMA}, segue-se as \textbf{ETAPAS
  VALORATIVAS}, analisando-se:

  \begin{itemize}
  \tightlist
  \item
    \textbf{Viabilidade}: pode ser eficazmente resolvido através da
    pesquisa?
  \item
    \textbf{Relevância}: é capaz de trazer conhecimentos novos?
  \item
    \textbf{Novidade}: está adequado ao estádio atual da evolução
    científica?
  \item
    \textbf{Exequibilidade}: om esse problema, é possível chegar a uma
    conclusão válida?
  \item
    \textbf{Oportunidade}: o problema atende a INTERESSES
    \textbf{particulares} e \textbf{gerais}?
  \end{itemize}
\end{itemize}

\hypertarget{forma-de-conceber-um-problema-cientuxedfico}{%
\paragraph{Forma de conceber um problema
científico}\label{forma-de-conceber-um-problema-cientuxedfico}}

\begin{itemize}
\tightlist
\item
  Qual o OBJETIVO DO TRABALHO DE PESQUISA ?

  \begin{itemize}
  \tightlist
  \item
    PROBLEMA de : estudo descritivo, de caráter informativO explicativo
    ou preditivo;
  \item
    PROBLEMA de \textbf{INFORMAÇÃO}: coleta de dados a respeito de
    estruturas e condutas observáveis, dentro de uma área de fenômenos;
  \item
    PROBLEMA de de \textbf{AÇÃO}: campos de ação onde determinados
    conhecimento sejam aplicados com êxito.
  \item
    INVESTIGAÇÃO \textbf{PURA} e \textbf{APLICADA}: estuda um problema
    relativo ao conhecimento científico ou à sua aplicabilidade.
  \end{itemize}
\end{itemize}

\begin{Shaded}
\begin{Highlighting}[]
\NormalTok{flowchart LR}
\NormalTok{A(A \textless{}b\textgreater{}Formulação do\textless{}br\textgreater{}PROBLEMA\textless{}/b\textgreater{}){-}{-}\textgreater{}B(Responde as PERGUNTAS){-}{-}\textgreater{}C(\textless{}b\textgreater{}O QUÊ\textless{}/b\textgreater{}?)}
\NormalTok{B{-}{-}\textgreater{}D(\textless{}b\textgreater{}COMO ?\textless{}/b\textgreater{})}
\end{Highlighting}
\end{Shaded}

\hypertarget{definiuxe7uxe3o-dos-termos}{%
\paragraph{Definição dos termos;}\label{definiuxe7uxe3o-dos-termos}}

\begin{itemize}
\tightlist
\item
  É tornar os TERMOS DA PESQUISA claros, compreensivos, objetivos e
  adequados que possam dar margem a interpretações errôneas;
\item
  O uso de termos apropriados, consistentemente definidos, CONTRIBUI
  para a \textbf{melhor compreensão da REALIDADE OBSERVADA};
\item
  Há TERMOS que precisam ser compreendidos com um \textbf{SIGNIFICADO
  ESPECÍFICO};
\item
  Há dois \textbf{TIPOS DE DEFINIÇÕES}:

  \begin{itemize}
  \tightlist
  \item
    \textbf{SIMPLES}: apenas traduzem o significado do termo ou
    expressão meno conhecida;
  \item
    \textbf{OPERACIONAL}: além do significado, ajudam, com exemplos, na
    compreensão do conceito, tornando clara a experiência no mundo
    extensional.
  \end{itemize}
\end{itemize}

\hypertarget{construuxe7uxe3o-de-hipuxf3teses}{%
\paragraph{Construção de
hipóteses;}\label{construuxe7uxe3o-de-hipuxf3teses}}

\hypertarget{indicauxe7uxe3o-de-variuxe1veis}{%
\paragraph{Indicação de
variáveis;}\label{indicauxe7uxe3o-de-variuxe1veis}}

\hypertarget{delimitauxe7uxe3o-da-pesquisa}{%
\paragraph{Delimitação da
pesquisa;}\label{delimitauxe7uxe3o-da-pesquisa}}

\hypertarget{amostragem}{%
\paragraph{Amostragem;}\label{amostragem}}

\hypertarget{seleuxe7uxe3o-de-muxe9todos-e-tuxe9cnicas}{%
\paragraph{Seleção de métodos e
técnicas;}\label{seleuxe7uxe3o-de-muxe9todos-e-tuxe9cnicas}}

\hypertarget{organizauxe7uxe3o-do-instrumental-de-pesquisa}{%
\paragraph{Organização do instrumental de
pesquisa;}\label{organizauxe7uxe3o-do-instrumental-de-pesquisa}}

\hypertarget{teste-de-instrumentos-e-procedimentos}{%
\subsubsection{Teste de instrumentos e
procedimentos}\label{teste-de-instrumentos-e-procedimentos}}

\begin{itemize}
\tightlist
\item
\end{itemize}

\backmatter
\end{document}
